%% Placeholder for introductory chapter

This class will introduce you to the fundamental theory and models of
optimization as well as the geometry that underlies them.  The first
portion of the course focuses on geometry: recalling and generalizing
linear algebraic concepts you first met in your linear algebra course.
The second portion focuses on optimization.  Presentation of
applications is woven throughout.  We will draw examples from diverse
areas of the engineering and natural sciences.  The material covered
in this course will prove of interest to students from all areas of
engineering, from the computer sciences and, more generally, from
disciplines wherein mathematical structure and the use of numerical
data is of central importance.

The main prior courses that we will be building on are vector
calculus and linear algebra.  No prior exposure to optimization is assumed.

The course text is {\em Optimization Models}, by G.~Calafiore and
L.~El Ghaoui, Cambridge Univ.~Press, 2014.  These notes are provided
as a supplement to, and not a replacement for, the course text.  Many problem set problems will be drawn from the course text.

\section*{Notation}

We work mainly with finite-dimensional real-valued vectors in the
course.  Lower-case is used for vectors.  A length-$n$ real vector $x$
is an ordered collection of real numbers where the $i$th coordinate of
$x$ is denoted $x_i \in \reals$. The default will be column vectors
so
\begin{equation*}
x = \left[ \begin{array}{c} x_1 \\ x_2 \\ \vdots \\ x_n\end{array} \right].
\end{equation*}
The length $n$ of the vector is also termed the ''dimension'' of the vector, which will subsequently be defined formally.  Alternately, the elements of $x$ may be complex, i.e., $x_i \in \complex$, or in some other field, $x_i \in \field$.  Again, our focus will be in the reals and we compactly denote the space of $x$ as $x \in \reals^n$.
The transpose of a column vector is a row vector. The transpose $x^\trans$ of $x$ is
\begin{equation*}
x^\trans = [ x_1 \ x_2 \ \ldots x_n].
\end{equation*}

We often need to work with a set (or a list) of vectors,
\begin{equation*}
\{ \vecx{1}, \vecx{2}, \ldots, \vecx{m}\}
\end{equation*}
where $\vecx{i} \in \reals^n$, $i \in \{1, 2, \ldots, m\}$ and
$(\vecx{i})^\trans = [ \vecx[1]{i} \ \vecx[2]{i} \ \ldots \ \vecx[n]{i} ] $.  The
set $\{1, 2, \ldots, m\}$ is the index set of $m$ elements.  We often
use the shorthand $[m]$ for the index set; in the above we would have
written $i \in [m]$.  We note that the book is not one hundred percent
consistent on this notation.  It sometimes reverts to the (simpler)
notation $\{x_1, x_2, \ldots x_m\}$ where $x_i \in \reals^n$ and $i
\in [m]$ for sets of vectors.  This less burdensome notation is used n
settings where sets of vectors are considered, but it is not necessary
also to index individual elements of the vectors.

Uppercase is used for matrices.  A matrix $A$ consisting of $n$ rows
and $m$ columns of real numbers is denoted $A \in \reals^{n \times
  m}$.  The element in the $i$th row and $j$th column of $A$ is
denoted $[A]_{ij}$ (alternately $a_{ij}$).  The transpose of $A$,
$A^\trans$ is the matrix the element in the $i$th row and $j$th
column of which is $[A]_{ji}$ (alternately $a_{ji}$).

Sets are denoted using calligraphic font.  (I will say ``script'' in
class since ``calligraphic'' is a mouthful.)  For example, the set of
vectors described above might be denoted $\calX = \{ \vecx{1}, \vecx{2},
\ldots, \vecx{m}\}$.  The cardinality of the set $\calX$ is denoted $|\calX|$;
in the above example $|\calX| = m$.  For some special sets we make an
exception.  In particular to denote real numbers, complex numbers, and
integers we respectively write $\reals$, $\complex$, and $\integers$.
Occasionally we have need to refer to the sets of non-negative and
positive real numbers, respectively denoted $\reals_+$ and
$\reals_{++}$.

Functions map elements of one set to another.  As with vectors we use
lowercase letters to denote functions.  While we typically use letters
towards the end of the Latin alphabet for vectors ($u$, $v$, $w$, $x$,
$y$, $z$), we typically use letters earlier in the alphabet for
functions ($f$, $g$, $h$), and letters in the middle for indexing
($i$, $j$, $k$, $l$, $m$, $n$).  

We write $f: \calX \rightarrow \calY$ to denote a function $f$ that
maps elements of $\calX$ to elements of $\calY$.  This notation is
akin to strongly-typed programming languages.  The function $f$ needs
an input in $\calX$ to be able to process it.  Elements not in $\calX$
are not acceptable as inputs.  That said, not every element of $\calX$
may be acceptable to $f$.  (E.g., if $f$ calculates the average age of
students in a class, no age inputted into the function should be
negative.)  The acceptable subset of $\calX$ is the domain of $f$,
denoted ${\rm dom} f$.  It is often convenient to define $f(x)
= \infty$ for all $x \notin {\rm dom} f$.  In that case ${\rm dom} f
= \{x \in \calX | \, |f(x)| < \infty\}$.  In this course we mostly
consider functions of the form $f : \reals^n \rightarrow
\reals^m$.  Some terminology that you might be aware of concerns the relationship
between $n$ and $m$.  If $n \neq m$ then $f$ is  a ``map''.
If $n = m$ then $f$ is an ``operator''.  If $m = 1$ then $f$ is a
``functional''.  An example of an $f: \reals \rightarrow \reals$ where ${\rm dom} f = \reals_+$ is plotted in Fig.~\ref{fig.oneDfunc}.


\begin{marginfigure}
  \centering
  \resizebox{7.5cm}{3cm}{
  %\begin{tikzpicture}[domain=0:10,samples = 200]
\begin{tikzpicture}[domain=0:10,samples = 60]
	  \draw[very thin,color=gray] (0,4) ;
%	  \draw[->] (-1,0) -- (10,0) node[font=\fontsize{120}{144},right]{$t$};
%	  \draw[->] (0,-2) -- (0,3) node[font=\fontsize{120}{144},above] {$x(t)$};
	  \draw[->] (-1,0) -- (10,0) node[font=\normalsize,right]{$\reals$};
	  \draw[->] (0,-2) -- (0,3) node[font=\normalsize,above] {$f$};
	  \draw[color=blue]   plot (\x,{(3+0.3517*sin(0.0759*3* \x r) +
		  0.8308*sin(0.0540*3* \x r) +0.5853*sin(0.5308*3* \x r) +
		  0.5497*sin(0.7792*3* \x r) +0.9172*sin(0.9340*3* \x r) +
		  0.2858*sin(0.1299*3* \x r) +0.7572*sin(0.5688*3* \x r) +
		  0.7537*sin(0.4694*3* \x r) +0.3804*sin(0.0119*3* \x r) +
		  0.5678*sin(0.3371*3* \x r) -(0.3517*cos(0.0759*3* \x r) +
		  0.8308*cos(0.0540*3* \x r) +0.5853*cos(0.5308*3* \x r) +
		  0.5497*cos(0.7792*3* \x r) +0.9172*cos(0.9340*3* \x r) +
		  0.2858*cos(0.1299*3* \x r) +0.7572*cos(0.5688*3* \x r) +
		  0.7537*cos(0.4694*3* \x r) +0.3804*cos(0.0119*3* \x r) +
		  0.5678*cos(0.3371*3* \x r)))/2 }) ;
  \end{tikzpicture}

  }
  \caption{A function $f: \reals \rightarrow \reals$.}
  \label{fig.oneDfunc}
\end{marginfigure}

