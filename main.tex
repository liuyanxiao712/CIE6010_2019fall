%%%%%%%%%%%%%%%%%%%%%%%%%%%%%%%%%%%%%%%%%
% Tufte-Style Book (Documentation Template)
% LaTeX Template
% Version 1.0 (5/1/13)
%
% This template has been downloaded from:
% http://www.LaTeXTemplates.com
%
% Original author:
% The Tufte-LaTeX Developers (tufte-latex.googlecode.com)
%
% License:
% Apache License (Version 2.0)
%
% IMPORTANT NOTE:
% In addition to running BibTeX to compile the reference list from the .bib
% file, you will need to run MakeIndex to compile the index at the end of the
% document.
%
%%%%%%%%%%%%%%%%%%%%%%%%%%%%%%%%%%%%%%%%%

%----------------------------------------------------------------------------------------
%	PACKAGES AND OTHER DOCUMENT CONFIGURATIONS
%----------------------------------------------------------------------------------------

\documentclass{tufte-book} % Use the tufte-book class which in turn uses the tufte-common class

\hypersetup{colorlinks} % Comment this line if you don't wish to have colored links

\usepackage{microtype} % Improves character and word spacing

\usepackage{lipsum} % Inserts dummy text

\usepackage{booktabs} % Better horizontal rules in tables

\usepackage{graphicx} % Needed to insert images into the document
%\graphicspath{{graphics/}} % Sets the default location of pictures
\graphicspath{{../figures/}}
\setkeys{Gin}{width=\linewidth,totalheight=\textheight,keepaspectratio} % Improves figure scaling

\usepackage{fancyvrb} % Allows customization of verbatim environments
\fvset{fontsize=\normalsize} % The font size of all verbatim text can be changed here

\newcommand{\hangp}[1]{\makebox[0pt][r]{(}#1\makebox[0pt][l]{)}} % New command to create parentheses around text in tables which take up no horizontal space - this improves column spacing
\newcommand{\hangstar}{\makebox[0pt][l]{*}} % New command to create asterisks in tables which take up no horizontal space - this improves column spacing

\usepackage{xspace} % Used for printing a trailing space better than using a tilde (~) using the \xspace command

\newcommand{\monthyear}{\ifcase\month\or January\or February\or March\or April\or May\or June\or July\or August\or September\or October\or November\or December\fi\space\number\year} % A command to print the current month and year

\newcommand{\openepigraph}[2]{ % This block sets up a command for printing an epigraph with 2 arguments - the quote and the author
\begin{fullwidth}
\sffamily\large
\begin{doublespace}
\noindent\allcaps{#1}\\ % The quote
\noindent\allcaps{#2} % The author
\end{doublespace}
\end{fullwidth}
}

\newcommand{\blankpage}{\newpage\hbox{}\thispagestyle{empty}\newpage} % Command to insert a blank page

\usepackage{units} % Used for printing standard units

\newcommand{\hlred}[1]{\textcolor{Maroon}{#1}} % Print text in maroon
\newcommand{\hangleft}[1]{\makebox[0pt][r]{#1}} % Used for printing commands in the index, moves the slash left so the command name aligns with the rest of the text in the index 
\newcommand{\hairsp}{\hspace{1pt}} % Command to print a very short space
\newcommand{\ie}{\textit{i.\hairsp{}e.}\xspace} % Command to print i.e.
\newcommand{\eg}{\textit{e.\hairsp{}g.}\xspace} % Command to print e.g.
\newcommand{\na}{\quad--} % Used in tables for N/A cells
\newcommand{\measure}[3]{#1/#2$\times$\unit[#3]{pc}} % Typesets the font size, leading, and measure in the form of: 10/12x26 pc.
\newcommand{\tuftebs}{\symbol{'134}} % Command to print a backslash in tt type in OT1/T1

\providecommand{\XeLaTeX}{X\lower.5ex\hbox{\kern-0.15em\reflectbox{E}}\kern-0.1em\LaTeX}
\newcommand{\tXeLaTeX}{\XeLaTeX\index{XeLaTeX@\protect\XeLaTeX}} % Command to print the XeLaTeX logo while simultaneously adding the position to the index

\newcommand{\doccmdnoindex}[2][]{\texttt{\tuftebs#2}} % Command to print a command in texttt with a backslash of tt type without inserting the command into the index

\newcommand{\doccmddef}[2][]{\hlred{\texttt{\tuftebs#2}}\label{cmd:#2}\ifthenelse{\isempty{#1}} % Command to define a command in red and add it to the index
{ % If no package is specified, add the command to the index
\index{#2 command@\protect\hangleft{\texttt{\tuftebs}}\texttt{#2}}% Command name
}
{ % If a package is also specified as a second argument, add the command and package to the index
\index{#2 command@\protect\hangleft{\texttt{\tuftebs}}\texttt{#2} (\texttt{#1} package)}% Command name
\index{#1 package@\texttt{#1} package}\index{packages!#1@\texttt{#1}}% Package name
}}

\newcommand{\doccmd}[2][]{% Command to define a command and add it to the index
\texttt{\tuftebs#2}%
\ifthenelse{\isempty{#1}}% If no package is specified, add the command to the index
{%
\index{#2 command@\protect\hangleft{\texttt{\tuftebs}}\texttt{#2}}% Command name
}
{%
\index{#2 command@\protect\hangleft{\texttt{\tuftebs}}\texttt{#2} (\texttt{#1} package)}% Command name
\index{#1 package@\texttt{#1} package}\index{packages!#1@\texttt{#1}}% Package name
}}

% A bunch of new commands to print commands, arguments, environments, classes, etc within the text using the correct formatting
\newcommand{\docopt}[1]{\ensuremath{\langle}\textrm{\textit{#1}}\ensuremath{\rangle}}
\newcommand{\docarg}[1]{\textrm{\textit{#1}}}
\newenvironment{docspec}{\begin{quotation}\ttfamily\parskip0pt\parindent0pt\ignorespaces}{\end{quotation}}
\newcommand{\docenv}[1]{\texttt{#1}\index{#1 environment@\texttt{#1} environment}\index{environments!#1@\texttt{#1}}}
\newcommand{\docenvdef}[1]{\hlred{\texttt{#1}}\label{env:#1}\index{#1 environment@\texttt{#1} environment}\index{environments!#1@\texttt{#1}}}
\newcommand{\docpkg}[1]{\texttt{#1}\index{#1 package@\texttt{#1} package}\index{packages!#1@\texttt{#1}}}
\newcommand{\doccls}[1]{\texttt{#1}}
\newcommand{\docclsopt}[1]{\texttt{#1}\index{#1 class option@\texttt{#1} class option}\index{class options!#1@\texttt{#1}}}
\newcommand{\docclsoptdef}[1]{\hlred{\texttt{#1}}\label{clsopt:#1}\index{#1 class option@\texttt{#1} class option}\index{class options!#1@\texttt{#1}}}
\newcommand{\docmsg}[2]{\bigskip\begin{fullwidth}\noindent\ttfamily#1\end{fullwidth}\medskip\par\noindent#2}
\newcommand{\docfilehook}[2]{\texttt{#1}\index{file hooks!#2}\index{#1@\texttt{#1}}}
\newcommand{\doccounter}[1]{\texttt{#1}\index{#1 counter@\texttt{#1} counter}}

\usepackage{makeidx} % Used to generate the index
\makeindex % Generate the index which is printed at the end of the document

% This block contains a number of shortcuts used throughout the book
\newcommand{\vdqi}{\textit{VDQI}\xspace}
\newcommand{\ei}{\textit{EI}\xspace}
\newcommand{\ve}{\textit{VE}\xspace}
\newcommand{\be}{\textit{BE}\xspace}
\newcommand{\VDQI}{\textit{The Visual Display of Quantitative Information}\xspace}
\newcommand{\EI}{\textit{Envisioning Information}\xspace}
\newcommand{\VE}{\textit{Visual Explanations}\xspace}
\newcommand{\BE}{\textit{Beautiful Evidence}\xspace}
\newcommand{\TL}{Tufte-\LaTeX\xspace}

%%%%%%%%%%%%%%%%%%%%%%%%%%%%%%%%%%%%%%%%%%%%%%%%%%%%%%%%
%%%%%%%%%%%%%%%%%%%%%%%%%%%%%%%%%%%%%%%%%%%%%%%%%%%%%%%%

\setcounter{secnumdepth}{1}
\usepackage{amsthm}

\usepackage{amsmath}
\usepackage{amssymb}
\usepackage{amsfonts}
\usepackage{graphicx}
\usepackage{epsfig}
\usepackage{graphics,eepic,epic,psfrag}
\usepackage{color}
\usepackage{appendix}
\usepackage{pgfplots}
\usepackage{tikz}
\usepackage{circuitikz}
%\usepackage{showframe}
\usepackage{gensymb}
\usetikzlibrary{arrows,arrows.meta,calc,quotes,angles}
\usepackage{verbatim}
\usepackage{enumitem}
\usepackage{textcomp} %includes copyright symbol
%\usepackage{mathabx} %% Has circular convolution symbol, but inclusion of this package causes a collision in definition of \degree
%\usepackage{stackengine}
\usepackage{ifthen}
\newboolean{tufteStyle}
\setboolean{tufteStyle}{true}

\newcommand{\ifTufte}{\ifthenelse{\boolean{tufteStyle}}}

%\setlength{\headheight}{30pt}
\headsep = 0.1in
\footskip = 0.35in
%\textheight = 8in
%\textwidth = 6.5in
%\hoffset = 0in
%\marginparwidth = 0in
%\marginparsep = 0in


\newcommand{\integers}{\mathbb{Z}}
\newcommand{\reals}{\mathbb{R}}
\newcommand{\rationals}{\mathbb{Q}}
\newcommand{\complex}{\mathbb{C}}
\newcommand{\field}{\mathbb{F}}
\newcommand{\indicator}{\mathbbm{1}}
\newcommand{\zeromat}{\mathbf{0}}

\newcommand{\trans}{{\rm T}}

% Note the #1 argument (coordinate index) is optional and defaults to null
\newcommand{\vecu}[2][]{u^{(#2)}_{#1}}  
\newcommand{\vecv}[2][]{v^{(#2)}_{#1}}  
\newcommand{\vecw}[2][]{w^{(#2)}_{#1}}  
\newcommand{\vecx}[2][]{x^{(#2)}_{#1}}  
\newcommand{\vecy}[2][]{y^{(#2)}_{#1}}  
\newcommand{\vecz}[2][]{z^{(#2)}_{#1}}  

\newcommand{\rvx}{X}
\newcommand{\svx}{x}
\newcommand{\rvy}{Y}
\newcommand{\svy}{y}
\newcommand{\rvz}{Z}
\newcommand{\svz}{z}
\newcommand{\rvs}{S}
\newcommand{\svs}{s}
\newcommand{\rvu}{U}
\newcommand{\svu}{u}
\newcommand{\rvv}{V}
\newcommand{\svv}{v}
\newcommand{\rvt}{T}
\newcommand{\svt}{t}


\newcommand{\calA}{{\cal A}}
\newcommand{\calB}{{\cal B}}
\newcommand{\calC}{{\cal C}}
\newcommand{\calE}{{\cal E}}
\newcommand{\calD}{{\cal D}}
\newcommand{\calH}{{\cal H}}
\newcommand{\calI}{{\cal I}}
\newcommand{\calJ}{{\cal J}}
\newcommand{\calN}{{\cal N}}
\newcommand{\calR}{{\cal R}}
\newcommand{\calS}{{\cal S}}
\newcommand{\calT}{{\cal T}}
\newcommand{\calU}{{\cal U}}
\newcommand{\calV}{{\cal V}}
\newcommand{\calW}{{\cal W}}
\newcommand{\calX}{{\cal X}}
\newcommand{\calY}{{\cal Y}}
\newcommand{\calZ}{{\cal Z}}

\newcommand{\proj}[2]{\text{Proj}_{#1}(#2)}
\newcommand{\spanvec}{\text{span}}
\newcommand{\Lone}[1]{\lVert #1 \rVert_1}
\newcommand{\Ltwo}[1]{\lVert #1 \rVert_2}
\newcommand{\Linf}[1]{\lVert #1 \rVert_\infty}


\newcommand{\E}[1]{E\left[{#1}\right]}
\DeclareMathOperator*{\argmax}{arg\,max}
\DeclareMathOperator*{\argmin}{arg\,min}
\DeclareMathOperator*{\argsup}{arg\,sup}
\DeclareMathOperator*{\arginf}{arg\,inf}
\DeclareMathOperator{\rank}{rank}
\DeclareMathOperator{\diag}{diag}
\DeclareMathOperator{\trace}{trace}
\DeclareMathOperator{\sgn}{sgn}

\newtheorem{defn}{Definition}


%\lstset{basicstyle=\footnotesize\ttfamily, breaklines=true}

%\lstset{upquote=true,showstringspaces=false}


\usepackage{./styles/optidef}
\newcommand{\tcr}{\textcolor{red}}
\newcommand{\tcb}{\textcolor{blue}}


\newtheorem{theorem}{Theorem}[chapter]
\newtheorem{lemma}[theorem]{Lemma}

\theoremstyle{definition}
\newtheorem{definition}[theorem]{Definition}
\newtheorem{example}[theorem]{Example}
\newtheorem{xca}[theorem]{Exercise}
\newtheorem{property}[theorem]{Property}

\theoremstyle{remark}
\newtheorem{remark}[theorem]{Remark}

\numberwithin{section}{chapter}
\numberwithin{equation}{chapter}
\numberwithin{figure}{chapter}

%%% Inputs for problem sets
%\input{./pathToProb.tex}
\newcommand{\RNum}[1]{\uppercase\expandafter{\romannumeral #1\relax}}  %% Roman Numerals
\newcommand{\subProbLabel}{\roman}
\newcommand{\subProbLabelB}{\alph}
\newcommand{\subProbLabelC}{\arabic}

%----------------------------------------------------------------------------------------
%	BOOK META-INFORMATION
%----------------------------------------------------------------------------------------

\title{Course Notes: \vspace{1ex} \\ \noindent Optimization theory and algorithms} % Title of the book

\author[Stark Draper]{Stark Draper} % Author

\publisher{Course notes: Version 1.03} % Publisher

%----------------------------------------------------------------------------------------

\begin{document}

\frontmatter


%----------------------------------------------------------------------------------------
%	EPIGRAPH
%----------------------------------------------------------------------------------------

%\thispagestyle{empty}
%\openepigraph{The public is more familiar with bad design than good design. It is, in effect, conditioned to prefer bad design, because that is what it lives with. The new becomes threatening, the old reassuring.}{Paul Rand, {\itshape Design, Form, and Chaos}}
%\vfill
%\openepigraph{A designer knows that he has achieved perfection not when there is nothing left to add, but when there is nothing left to take away.}{Antoine de Saint-Exup\'{e}ry}
%\vfill
%\openepigraph{\ldots the designer of a new system must not only be the implementor and the first large-scale user; the designer should also write the first user manual\ldots If I had not participated fully in all these activities, literally hundreds of improvements would never have been made, because I would never have thought of them or perceived why they were important.}{Donald E. Knuth}

%----------------------------------------------------------------------------------------

\maketitle % Print the title page

%----------------------------------------------------------------------------------------
%	COPYRIGHT PAGE
%----------------------------------------------------------------------------------------

\newpage
\begin{fullwidth}
~\vfill
\thispagestyle{empty}
\setlength{\parindent}{0pt}
\setlength{\parskip}{\baselineskip}
Copyright \copyright\ \the\year\ \thanklessauthor

%\par\smallcaps{Published by \thanklesspublisher}

%\par\smallcaps{tufte-latex.googlecode.com}

%\par Licensed under the Apache License, Version 2.0 (the ``License''); you may not use this file except in compliance with the License. You may obtain a copy of the License at \url{http://www.apache.org/licenses/LICENSE-2.0}. Unless required by applicable law or agreed to in writing, software distributed under the License is distributed on an \smallcaps{``AS IS'' BASIS, WITHOUT WARRANTIES OR CONDITIONS OF ANY KIND}, either express or implied. See the License for the specific language governing permissions and limitations under the License.\index{license}

%\par\textit{First printing, \monthyear}
\par\textit{\monthyear}
\end{fullwidth}

%----------------------------------------------------------------------------------------

\tableofcontents % Print the table of contents

%----------------------------------------------------------------------------------------

%%% SCD: commented out on 22 Sept 2019, pretty minimal and unhelpful for now
%%%\listoffigures % Print a list of figures

%----------------------------------------------------------------------------------------

%%% SCD: commented out on 22 Sept 2019, pretty minimal and unhelpful for now
%%%\listoftables % Print a list of tables

%----------------------------------------------------------------------------------------
%	DEDICATION PAGE
%----------------------------------------------------------------------------------------
\iffalse
\cleardoublepage
~\vfill
\begin{doublespace}
\noindent\fontsize{18}{22}\selectfont\itshape
\nohyphenation
Dedicated to those who appreciate \LaTeX{} and the work of \mbox{Edward R.~Tufte} and \mbox{Donald E.~Knuth}.
\end{doublespace}
\vfill
\vfill
\fi
%----------------------------------------------------------------------------------------
%	INTRODUCTION
%----------------------------------------------------------------------------------------

%\cleardoublepage
%\chapter*{Introduction} % The asterisk leaves out this chapter from the table of contents

%This sample book discusses the design of Edward Tufte's books\cite{Tufte2001,Tufte1990,Tufte1997,Tufte2006} and the use of the \doccls{tufte-book} and \doccls{tufte-handout} document classes.

%----------------------------------------------------------------------------------------

\newpage 
\vspace*{0.2in}
\centerline{\bf \large Remarks, feedback, and versions}\vspace{2ex}

\noindent These notes are in development in fall term \the\year.
These notes are meant to complement, and not replace, the course text.
They indicate to the reader our specific trajectory through the text
and the emphasis of material in our course.  The majority of thanks
for this teaching resource are due to Zhipeng Huang and Yanxiao Liu
who built up these notes from scratch.  Thank you Zhipeng and Yanxiao!
As we progress through the semester updated versions with additional
chapters and edits will be distributed.  The main differences between
distributions are noted below.  Corrections of typos and errors, and
other suggestions are welcome and appreciated.  Please email any such
comments to \texttt{eceCourseProfDraper@gmail.com}.  Please include
the course number, and the notes version number,  in the subject line of your message, as course notes
for distinct courses are in parallel development. \vspace{0.2in} \\

%%%% Winter 2019 distribution
\begin{tabbing}
\noindent \= Version 1.01: \ \= Initial distribution of chapters 1 and 2.\\
\noindent \= Version 1.02: \ \= Initial distribution of chapters 3 and 4.\\
\noindent \= Version 1.03: \ \= Initial distribution of chapter 5.\\
\end{tabbing}

\newpage
\mainmatter


\chapter{Introduction}
\label{ch.intro}
%% Placeholder for introductory chapter

This class will introduce you to the fundamental theory and models of
optimization as well as the geometry that underlies them.  The first
portion of the course focuses on geometry: recalling and generalizing
linear algebraic concepts you first met in your linear algebra course.
The second portion focuses on optimization.  Presentation of
applications is woven throughout.  We will draw examples from diverse
areas of the engineering and natural sciences.  The material covered
in this course will prove of interest to students from all areas of
engineering, from the computer sciences and, more generally, from
disciplines wherein mathematical structure and the use of numerical
data is of central importance.

The main prior courses that we will be building on are vector
calculus and linear algebra.  No prior exposure to optimization is assumed.

The course text is {\em Optimization Models}, by G.~Calafiore and
L.~El Ghaoui, Cambridge Univ.~Press, 2014.  These notes are provided
as a supplement to, and not a replacement for, the course text.  Many problem set problems will be drawn from the course text.

\section*{Notation}

We work mainly with finite-dimensional real-valued vectors in the
course.  Lower-case is used for vectors.  A length-$n$ real vector $x$
is an ordered collection of real numbers where the $i$th coordinate of
$x$ is denoted $x_i \in \reals$. The default will be column vectors
so
\begin{equation*}
x = \left[ \begin{array}{c} x_1 \\ x_2 \\ \vdots \\ x_n\end{array} \right].
\end{equation*}
The length $n$ of the vector is also termed the ''dimension'' of the vector, which will subsequently be defined formally.  Alternately, the elements of $x$ may be complex, i.e., $x_i \in \complex$, or in some other field, $x_i \in \field$.  Again, our focus will be in the reals and we compactly denote the space of $x$ as $x \in \reals^n$.
The transpose of a column vector is a row vector. The transpose $x^\trans$ of $x$ is
\begin{equation*}
x^\trans = [ x_1 \ x_2 \ \ldots x_n].
\end{equation*}

We often need to work with a set (or a list) of vectors,
\begin{equation*}
\{ \vecx{1}, \vecx{2}, \ldots, \vecx{m}\}
\end{equation*}
where $\vecx{i} \in \reals^n$, $i \in \{1, 2, \ldots, m\}$ and
$(\vecx{i})^\trans = [ \vecx[1]{i} \ \vecx[2]{i} \ \ldots \ \vecx[n]{i} ] $.  The
set $\{1, 2, \ldots, m\}$ is the index set of $m$ elements.  We often
use the shorthand $[m]$ for the index set; in the above we would have
written $i \in [m]$.  We note that the book is not one hundred percent
consistent on this notation.  It sometimes reverts to the (simpler)
notation $\{x_1, x_2, \ldots x_m\}$ where $x_i \in \reals^n$ and $i
\in [m]$ for sets of vectors.  This less burdensome notation is used n
settings where sets of vectors are considered, but it is not necessary
also to index individual elements of the vectors.

Uppercase is used for matrices.  A matrix $A$ consisting of $n$ rows
and $m$ columns of real numbers is denoted $A \in \reals^{n \times
  m}$.  The element in the $i$th row and $j$th column of $A$ is
denoted $[A]_{ij}$ (alternately $a_{ij}$).  The transpose of $A$,
$A^\trans$ is the matrix the element in the $i$th row and $j$th
column of which is $[A]_{ji}$ (alternately $a_{ji}$).

Sets are denoted using calligraphic font.  (I will say ``script'' in
class since ``calligraphic'' is a mouthful.)  For example, the set of
vectors described above might be denoted $\calX = \{ \vecx{1}, \vecx{2},
\ldots, \vecx{m}\}$.  The cardinality of the set $\calX$ is denoted $|\calX|$;
in the above example $|\calX| = m$.  For some special sets we make an
exception.  In particular to denote real numbers, complex numbers, and
integers we respectively write $\reals$, $\complex$, and $\integers$.
Occasionally we have need to refer to the sets of non-negative and
positive real numbers, respectively denoted $\reals_+$ and
$\reals_{++}$.

Functions map elements of one set to another.  As with vectors we use
lowercase letters to denote functions.  While we typically use letters
towards the end of the Latin alphabet for vectors ($u$, $v$, $w$, $x$,
$y$, $z$), we typically use letters earlier in the alphabet for
functions ($f$, $g$, $h$), and letters in the middle for indexing
($i$, $j$, $k$, $l$, $m$, $n$).  

We write $f: \calX \rightarrow \calY$ to denote a function $f$ that
maps elements of $\calX$ to elements of $\calY$.  This notation is
akin to strongly-typed programming languages.  The function $f$ needs
an input in $\calX$ to be able to process it.  Elements not in $\calX$
are not acceptable as inputs.  That said, not every element of $\calX$
may be acceptable to $f$.  (E.g., if $f$ calculates the average age of
students in a class, no age inputted into the function should be
negative.)  The acceptable subset of $\calX$ is the domain of $f$,
denoted ${\rm dom} f$.  It is often convenient to define $f(x)
= \infty$ for all $x \notin {\rm dom} f$.  In that case ${\rm dom} f
= \{x \in \calX | \, |f(x)| < \infty\}$.  In this course we mostly
consider functions of the form $f : \reals^n \rightarrow
\reals^m$.  Some terminology that you might be aware of concerns the relationship
between $n$ and $m$.  If $n \neq m$ then $f$ is  a ``map''.
If $n = m$ then $f$ is an ``operator''.  If $m = 1$ then $f$ is a
``functional''.  An example of an $f: \reals \rightarrow \reals$ where ${\rm dom} f = \reals_+$ is plotted in Fig.~\ref{fig.oneDfunc}.


\begin{marginfigure}
  \centering
  \resizebox{7.5cm}{3cm}{
  %\begin{tikzpicture}[domain=0:10,samples = 200]
\begin{tikzpicture}[domain=0:10,samples = 60]
	  \draw[very thin,color=gray] (0,4) ;
%	  \draw[->] (-1,0) -- (10,0) node[font=\fontsize{120}{144},right]{$t$};
%	  \draw[->] (0,-2) -- (0,3) node[font=\fontsize{120}{144},above] {$x(t)$};
	  \draw[->] (-1,0) -- (10,0) node[font=\normalsize,right]{$\reals$};
	  \draw[->] (0,-2) -- (0,3) node[font=\normalsize,above] {$f$};
	  \draw[color=blue]   plot (\x,{(3+0.3517*sin(0.0759*3* \x r) +
		  0.8308*sin(0.0540*3* \x r) +0.5853*sin(0.5308*3* \x r) +
		  0.5497*sin(0.7792*3* \x r) +0.9172*sin(0.9340*3* \x r) +
		  0.2858*sin(0.1299*3* \x r) +0.7572*sin(0.5688*3* \x r) +
		  0.7537*sin(0.4694*3* \x r) +0.3804*sin(0.0119*3* \x r) +
		  0.5678*sin(0.3371*3* \x r) -(0.3517*cos(0.0759*3* \x r) +
		  0.8308*cos(0.0540*3* \x r) +0.5853*cos(0.5308*3* \x r) +
		  0.5497*cos(0.7792*3* \x r) +0.9172*cos(0.9340*3* \x r) +
		  0.2858*cos(0.1299*3* \x r) +0.7572*cos(0.5688*3* \x r) +
		  0.7537*cos(0.4694*3* \x r) +0.3804*cos(0.0119*3* \x r) +
		  0.5678*cos(0.3371*3* \x r)))/2 }) ;
  \end{tikzpicture}

  }
  \caption{A function $f: \reals \rightarrow \reals$.}
  \label{fig.oneDfunc}
\end{marginfigure}



\chapter{Vectors and functions}
\label{ch.vecFunc}
%% Placeholder for chapter on vector and functions

\noindent\textcircled{1} Geometry\\
-Vectors and vector spaces\\
-Norms\\
-Inner product\\

\noindent\textcircled{2} Projection\\
-On to subspace\\
-On to affine sets\\
-Non-Euclidean\\

\noindent\textcircled{3} Functions\\
-Functions and sets\\
-Linear and affine\\
-Gradients and Taylor approximations\\

\newpage

As mentioned in introduction, the 1st part of this course will focus on geometry. Linear algebra is the mathematical study of geometry in (arbitrary large) dimensions.

It turns out that your geometry xxx from $\reals^{2}$ to $\reals^{3}$(planes and space) is extremely helpful to conceive of large dimensional sets and understand operations on them.

Lots(but not all) of what we cover in the first few topics will repeat what you saw in your linear algebra course(Math 188/185).

\vspace{0.5cm}
\noindent Why repeat?

-Linear algebra in year 1, perhaps semester 1.

-It takes time xxx, to "get" linear algebra. If you are in this course, you are likely to come up a lot going forward.

-In your linear algebra course may have concentrate more on the "algebra" side of linear algebra rather than the geometry side. Our focus will be on the latter. E.g., $y-x^{2}=0$ is an algebraic relation but defines a geometric object(a parabola).

-If you took ECE216 with me you will see some familiar examples, and I encourage you to  connecting questions (perhaps after class as xxx all students took ECE216).

\newpage

\noindent\textbf{Vector}:A collection of numbers.
\begin{equation*}
x = \left[ \begin{array}{c} x_1 \\ x_2 \\ \vdots \\ x_n\end{array} \right].
\end{equation*}
where each $x_i \in \reals$ or $x_i \in \complex$. The length $n$ of the vector is also termed the "dimension" of the vector, which will subsequently be defined formally.

Our default will be a column vector as we describe above. Transpose x yields a row vector,   
\begin{equation*}
x^\trans = [ x_1 \ x_2 \ \ldots x_n].
\end{equation*}
and occasionally write as a list $(x_{1}, x_{2},\cdots,x_{n})$.  Note that a vector is not a set of numbers since order matters.

Also, we often need to work with a set(or list) of vectors,
\begin{equation*}
\{ \vecx{1}, \vecx{2}, \ldots, \vecx{m}\}
\end{equation*}
where $\vecx{i} \in \reals^n$, $i \in \{1, 2, \ldots, m\}$, $i \in [m] = \{1,2,\cdots,m\}$ and
$(\vecx{i})^\trans = [ \vecx[1]{i} \ \vecx[2]{i} \ \ldots \ \vecx[n]{i} ]$.  

Note: book not 100\% consistent, XXX

\vspace{0.5cm}
\noindent\textbf{Vector Space}

To this point a vector is just a lost of numbers.

To get to geometry, we need to define how to add pairs of vectors and how to scale vectors.

Addition: $u=v^{1}+v^{2}$, means $u_{i}=v_{i}^{1}+v_{i}^{2}$ for all $i\in [n]$.

Scaling: $u=av$,  means $u_{i}=v_{i}^{1}+v_{i}^{2}$ for all $i\in [n]$.

Linear combination: $\sum_{i=1}^{m} a_{i}v^{(i)}$

\begin{marginfigure}
	\centering
	\resizebox{7.5cm}{3cm}{\begin{tikzpicture}
\draw [white, <->] (0,5) -- (0,0) -- (5,0);
\draw [-latex, ultra thick] (0,0) -- (0,1);
\draw [-latex, ultra thick] (0,0) -- (3,3);
\draw [-latex, ultra thick] (0,1) -- (3,3);
\node [above] at (0,1) {$v^{(1)}$};
\node [below] at (1.5,1.3) {$u$};
\node [above] at (3,3) {$v^{(2)}$};
\end{tikzpicture}}
	\caption{Add}
	\label{fig.2-1}
\end{marginfigure}
\begin{marginfigure}
	\centering
	\resizebox{7.5cm}{3cm}{\begin{tikzpicture}
\draw [white, <->] (0,5) -- (0,0) -- (5,0);
\draw [-latex, ultra thick] (0,0) -- (1.5,1.5);
\draw [-latex, ultra thick] (1.5,1.5) -- (3,3);
\node [above] at (0.75,0.75) {$v^{(1)}$};
\node [above] at (3,3) {$u=2v^{(1)}$};
\end{tikzpicture}}
	\caption{Scale}
	\label{fig.2-2}
\end{marginfigure}

Note that If $a=0$, then $u=av^{(1)}=0$.

\vspace{0.5cm}

With respect to the XXX at the origin can think of as xxx as a displacement(a move through the space) from the origin.

For any vector $v\in\reals$, $v=v-0$.

Note: XXX


Vector Space: a set of vectors that is closed under addition and scaling.

Formally we need following axioms:

Commutativity: $u+v=v+u$

Associativity: $(u+v)+w=u+(v+w)$

Distributivity: $a(u+v)=au+av$,\ $(a+b)u=au+bu$

Identity element of addition: $\exists 0\in \gamma$ s.t. $u+0=u$

Inverse elements of addition: $\exists -u\in \gamma$ s.t. $u+(-u)=0$

Identity element of scalar multiplication: $\exists a\in \reals\ \text{or}\ \complex$ s.t. $au=u$

In this course our focus is on $\reals^{n}$, i.e., finite-length vectors with real elements.

It is also useful to note that the geometric ideas could apply to lots of other spaces.\\
\textcircled{1} Finite-length complex vector\\
we need this esp for discussion of eigenvalues and eigenvectors. But also important examples in quantum computing.\\
\textcircled{2} $\infty$-length complex sequences(DT signals\&system)\\
\textcircled{3} Complex functions defined on real line(CT signals\&system)\\
\textcircled{4} Polynomials of degree at most n-1\\
$$P_{n-1}=\{P|p(t)=a_{n-1}t^{n-1}+a_{n-2}t^{n-2}+\cdots + a_{1}t+a_{0} \}$$
It can xxx linear combinations of polynomials and doesn't increase degree, so closed.\\
\textcircled{5} Sets of matrices(will discuss later)\\

Note: Some authors prefer “linear space" rather than "vector space" since elements of space are not always vectors in the sense of a list.

\vspace{0.5cm}

Span and subspace

If I give you a set of vectors, thinking of each as a displacement, anywhere you can get to via linear combinations is the "span" at the set.

If $S=\{v^{(1)}, v^{(2)}, \cdots ,v^{(m)}\}$, where each $v^{(i)}\in \reals^{n}$

Then span(S)$=\{\sum_{i=1}^{m} a_{i}v^{(i)}| a_{i}\in \reals ,\forall i\in [m]\}$

Example 1\\
Let $S=\{v^{(1)}\}=\left\{ 
\left[ 
\begin{array}{c} 
1 \\
1
\end{array}
\right]\right\}$

then
\begin{align*}
span(S)&=span({v^{(1)}})\\
&=\left\{\left[ 
	\begin{array}{c} 
	x \\
	y
	\end{array}
	\right]|x=y\right\}\\
&=\left\{a\left[ 
	\begin{array}{c} 
	1 \\
	1
	\end{array}
	\right]|a\in\reals\right\}
\end{align*}\\
\begin{marginfigure}
	\centering
	\resizebox{7.5cm}{3cm}{\begin{tikzpicture}
\draw [white, <->] (0,3) -- (0,0) -- (3,0);
\draw [-latex, ultra thick] (0,0) -- (1,1);
\draw [dashed,ultra thick] (-2,-2) -- (2,2);
\node [below right] at (0,0) {(0,0)};
\node [above left] at (1,1) {$v^{(1)}$};
\end{tikzpicture}}
	\caption{}
	\label{fig.2-3}
\end{marginfigure}


Example 2\\
$S=\{v^{(1)}, v^{(2)}\}=\left\{\left[ 
\begin{array}{c} 
1\\
1\\
0
\end{array}
\right],           
\left[ 
\begin{array}{c} 
1\\
-1\\
0
\end{array}
\right]
\right\}$

\begin{align*}
span(S)&=\{a_{1}v^{(1)}, a_{2}v^{(2)}|(a_{1}, a_{2})\in \reals_{2}\}\\
&=\left\{\left[ 
	\begin{array}{c} 
	x\\
	y\\
	0
	\end{array}
	\right] |x\in\reals , y\in\reals\right\}\\
&=\text{x-y plane}
\end{align*}\\
\begin{marginfigure}
	\centering
	\resizebox{7.5cm}{3cm}{\begin{tikzpicture}
\draw [<->] (0,4) -- (0,0) -- (4,0);
\draw [->] (0,0) -- (3,2);
\draw [dashed] (0,0) -- (-1.5,-1);
\draw [-latex, ultra thick] (0,0) -- (1.7,0.5);
\draw [-latex,ultra thick] (0,0) -- (0.8,-1);
\node [below] at (4,0) {x};
\node [below] at (3,2) {y};
\node [left] at (0,4) {z};
\node [above right] at (1.7,0.5) {$v^{(1)}$};
\node [below left] at (0.8,-1) {$v^{(2)}$};
\end{tikzpicture}}
	\caption{}
	\label{fig.2-4}
\end{marginfigure}


In fact, the span of a set of vectors is a "subspace" is a subset of the XXX vector space($\reals^{n}$ or $\complex^{n}$). XXXX properties of a vector space

Note: $0\in\reals_{n}$ always included since we can set all coefficients $a_{i}=0$ for all $i$.

Subspace is a "flat" that goes through the origin.

\vspace{0.5cm}
\noindent\textbf{Linear independent set}

$S=\{v^{(1)},\cdots , v^{(n)}\}$ is a linear, independent set if no element of $S$ can be expressed as a linear combination of the others.

The set $S$ is linearly independent if the only $a_{i}$ that satisfies

$\sum_{i=1}^{m}a_{i}v^{(i)}=0$ is if $a_{i}=0$ $\forall i\in [m]$

If this were not true, letting $l\in[m]$ be s.t. $a_{l}\neq 0$ would have
$$a_{l}v^{l}+\sum_{i\neq l} a_{i}v^{(i)}=0$$
$$v^{(l)}=\sum_{i\neq l}(\frac{-a_{i}}{a_{l}})v^{(i)}$$

Note that $a_{l}\neq 0$. So it is not linearly independent.

\vspace{0.5cm}
\noindent Importance of linearly independent

For any $u\in span(S)$ that is a unique choice of the $a_{i}$ in the expression
$u=\sum_{i=1}^{m} a_{i}v^{(i)}$, i.e., only one way to express.

\begin{marginfigure}
	\centering
	\resizebox{7.5cm}{3cm}{\begin{tikzpicture}
\draw [<->] (0,4) -- (0,0) -- (4,0);
\draw [-latex, ultra thick] (0,0) -- (2,2);
\draw [-latex,ultra thick] (0,0) -- (2,0);
\node [above right] at (2,2) {$v^{(1)}$};
\node [below] at (2,0) {$v^{(2)}$};
\end{tikzpicture}}
	\caption{}
	\label{fig.2-4}
\end{marginfigure}
\begin{marginfigure}
	\centering
	\resizebox{7.5cm}{3cm}{\begin{tikzpicture}
\draw [<->] (0,5) -- (0,0) -- (5,0);
\draw [-latex, ultra thick] (0,0) -- (2,2);
\draw [-latex,ultra thick] (2,2) -- (4,2);
\draw [dashed,ultra thick] (-2,2) -- (5,2);
\node [below] at (5,0) {$u_{1}$};
\node [left] at (0,5) {$u_{2}$};
\node [above right] at (4,2) {$a_{1}v^{(1)}$};
\node [below right] at (2,2) {$a_{2}v^{(2)}$};
\end{tikzpicture}}
	\caption{}
	\label{fig.2-4}
\end{marginfigure}

$v^{(1)}= 
\left[ 
\begin{array}{c} 
1 \\
0
\end{array}
\right]$
,
$v^{(1)}= 
\left[ 
\begin{array}{c} 
1 \\
1
\end{array}
\right]$

No redundancy is representation.

observe: it did have redundancy in S. For example,

$S=\left\{\left[ 
\begin{array}{c} 
1\\
0
\end{array}
\right],           
\left[ 
\begin{array}{c} 
1\\
1
\end{array}
\right],
\left[ 
\begin{array}{c} 
0\\
1
\end{array}
\right]
\right\}$

can always shrink S by removing elements to get a linearly independent set. Such an irreducible or linearly independent set can serve as a basis for span(S).

Any largest linearly independent subset of $S=\{v^{(1)},\cdots , v^{(m)}\}$,$B=\{v^{(1)},\cdots , v^{(k)}\}$, $k\leq m$ is a basis for span(S), and the dimension of span(S), dim(span(S))=k.


Example

$v^{(1)}= 
\left[ 
\begin{array}{c} 
1 \\
1 \\
1
\end{array}
\right]$,
$v^{(2)}= 
\left[ 
\begin{array}{c} 
1 \\
2 \\
0
\end{array}
\right]$,
$v^{(3)}= 
\left[ 
\begin{array}{c} 
1 \\
3 \\
1
\end{array}
\right]$

an linearly independent spanning set form a basis for

$span\left(\{v^{(1)},v^{(2)}, v^{(3)}\right\}=\reals^{3}$

But, if swap $v^{(3)}= 
\left[ 
\begin{array}{c} 
3 \\
4 \\
2
\end{array}
\right]
=2v^{(1)}+v^{(2)}$

Then $\left(v^{(1)},v^{(2)}, v^{(3)}\right)$ is not a basis, and it need to be reduced to:

$span\left(\{v^{(1)},v^{(2)}\}\right)=span\left(\{v^{(1)},v^{(2)}\}\right)=span\left(\{v^{(1)},v^{(2)}\}\right)=span(S)$

We can prove that each a basis for span(S) all have same coordinates.

Example

Perhaps most familiar basis is the "standard" basis

$v^{(1)}= 
\left[ 
\begin{array}{c} 
1 \\
0 \\
0 \\
\vdots
\end{array}
\right]$,
$v^{(2)}= 
\left[ 
\begin{array}{c} 
0 \\
1 \\
0 \\
\vdots
\end{array}
\right]$, $\cdots$,
$v^{(n)}= 
\left[ 
\begin{array}{c} 
0 \\
0 \\
\vdots \\
1 
\end{array}
\right]$


often see "e" for standard basis, i.e., $e^{(i)}=v^{(i)}$.

Note that our book uses $e_{i}$ for $e^{(i)}$.

\vspace{0.5cm}
\noindent\textbf{Norms}: Another important property, idea of distance on length

Familiar: Euclidean distance 

But not only notion of distance. E.g. walking through downtown Toronto or NYC. Blocks you walk along the shortest park.

figure here

So, multiple sense of distance, a sense of distance for a vector is a "norm", some properties on norm must satisfy.

A norm $\Vert\cdot\Vert$ is a function such that $\Vert\cdot\Vert : \gamma\mapsto\reals$
and satisfies

(a)$\Vert v \Vert\geq 0$, $\forall v\in \gamma$, and $\Vert v \Vert=0$ iff $v=0$.

(b)$\Vert u+v \Vert\leq \Vert u \Vert+\Vert v \Vert$, $\forall u, v\in \gamma$.

(c)$\Vert au \Vert= \vert a\vert\Vert u \Vert$, $\forall a\in\reals, u\in\gamma$

Note:$\gamma$ can be either $\reals$ or $\complex$, if $\gamma\in\complex$ we should have $a\in\complex$ in (c).

a family of norms but will come up after are:

$L_{p}$ norm:

$$\Vert x\Vert_{p}=(\sum_{k=1}^{n}\vert x_{k}\vert^{p})^{1/p}, 1\leq p\leq\infty$$

$L_{2}$ norm: Euclidean length

$$\Vert x\Vert_{2}=\sqrt{\sum_{k=1}^{n}\vert x_{k}\vert^{2}}$$

$L_{1}$ norm:

$$\Vert x\Vert_{1}=\sum_{k=1}^{n}\vert x_{k}\vert$$

$L_{\infty}$ norm:

$$\Vert x\Vert_{\infty}=\lim_{p\to\infty}\Vert x_{k}\Vert_{p}=\max_{k\in [n]} \vert x_{k}\vert$$


Length is a notion of "size"

A natural notion of its "size" of a set is the number of non zero XXX

i.e., carnality of non-zero support

$$card(x)=\sum_{k=1}^{n}\mathbb{1}_{x_{k}\neq 0}$$

Sometimes it is called "$L_{0}$" $norm \Vert x\Vert_{0}$ since

$$card(x)=\lim_{p\to 0}(\sum_{k=1}^{n}\vert x_{k}\vert^{p})^{p}$$

But not a norm (so this terminology is inaccurate). E.g., it doesn't satisfy property (c),

$$card(2x)=card(x)\neq 2card(x)$$


To visualize a norm we often plot it unit norm-ball

$$\beta_{p}=\{x|\Vert x\Vert_{p}\leq 1\}$$

$L_{2}$

\begin{marginfigure}
	\centering
	\resizebox{7.5cm}{3cm}{%% page 15
%% fig 1
\begin{tikzpicture}
\draw[thick,-stealth] (-2, 0) -- (2, 0);
\draw[thick,-stealth] (0, -2) -- (0, 2);

\filldraw[fill=gray, draw=black, opacity=0.5] (0,0) circle [radius=1];
\node [below right] at (1, 0) {1};
\node [above right] at (0, 1) {1};
\end{tikzpicture}}
	\caption{}
	\label{}
\end{marginfigure}

$L_{1}$ : $\{x|\vert x_{1}\vert\leq 1\}$

\begin{marginfigure}
	\centering
	\resizebox{7.5cm}{3cm}{%% page 15
%% fig 2
\begin{tikzpicture}
\draw[thick,-stealth] (-2, 0) -- (2, 0);
\draw[thick,-stealth] (0, -2) -- (0, 2);

\draw[dashed] (-1, -1) rectangle (1, 1);
\filldraw[draw=black, fill=gray, opacity=0.5] (-1, 0) -- (0, 1) -- (1, 0) -- (0, -1) -- (-1, 0);
\node [below right] at (1, 0) {1};
\node [above right] at (0, 1) {1};
\end{tikzpicture}
}
	\caption{}
	\label{}
\end{marginfigure}

(a) First see inside the box, clearly $\vert x_{1}\vert\leq 1$ and $\vert x_{2}\vert\leq 1$

(b) Look at the position we want, $x_{1}+x_{2}\leq 1$, i.e., $x_{2}\leq 1-x_{1}$

(c) Rest by symmetry


$L_{\infty}$ : $\{x|\max\{\vert x_{1}\vert, \vert x_{2}\vert\} \leq 1\}$

\begin{marginfigure}
	\centering
	\resizebox{7.5cm}{3cm}{%% page 15
%% fig 3
\begin{tikzpicture}
\draw[thick,-stealth] (-2, 0) -- (2, 0);
\draw[thick,-stealth] (0, -2) -- (0, 2);

\filldraw[fill=gray, draw=black, opacity=0.5] (-1, -1) rectangle (1, 1);
\node [below right] at (1, 0) {1};
\node [above right] at (0, 1) {1};
\end{tikzpicture}
}
	\caption{}
	\label{}
\end{marginfigure}


what about "$L_{0}$"? $\{x|card(x)\leq 1\}$

\begin{figure}
	\centering
	\resizebox{7.5cm}{3cm}{%% page 15
%% fig 4
\begin{tikzpicture}
\draw[thick,-stealth] (-2, 0) -- (2, 0);
\draw[thick,-stealth] (0, -2) -- (0, 2);

\draw (-1, 0) -- (1, 0)node[below]{1};
\draw (0, -1) -- (0, 1)node[right]{1};

\fill (0, 1) circle (1pt);
\fill (1, 0) circle (1pt);
\fill (0, -1) circle (1pt);
\fill (-1, 0) circle (1pt);
\end{tikzpicture}}
	\caption{}
	\label{}
\end{figure}

Not much of a "ball"


To visualize a bit more look at "level sets" of the norm balls.

$\{x|\vert x\vert=c \}$, see for $c=\frac{1}{2}, 1, 2$

$L_{1}$

\begin{marginfigure}
	\centering
	\resizebox{7.5cm}{3cm}{%% page 16
%% fig 1
\begin{tikzpicture}
\draw[thick,-stealth] (-2, 0) -- (2, 0);
\draw[thick,-stealth] (0, -2) -- (0, 2);

\draw (0,0) circle [radius=0.7];
\node [below left] at (0.7, 0) {$\frac{1}{\sqrt 2}$};

\draw (0,0) circle [radius=1];
\node [below right] at (1, 0) {1};

\draw (0,0) circle [radius=1.44];
\node [below right] at (1.44, 0) {$\sqrt 2$};

\end{tikzpicture}
}
	\caption{}
	\label{}
\end{marginfigure}

$L_{2}$

\begin{marginfigure}
	\centering
	\resizebox{7.5cm}{3cm}{%% page 16
%% fig 2
\begin{tikzpicture}
\draw[thick,-stealth] (-2, 0) -- (2, 0);
\draw[thick,-stealth] (0, -2) -- (0, 2);

\draw (-0.5, 0) -- (0, 0.5) -- (0.5, 0) -- (0, -0.5) -- (-0.5, 0);
\node [below right] at (0.5, 0) {0.5};

\draw (-1, 0) -- (0, 1) -- (1, 0) -- (0, -1) -- (-1, 0);
\node [below right] at (1, 0) {1};

\draw (-1.5, 0) -- (0, 1.5) -- (1.5, 0) -- (0, -1.5) -- (-1.5, 0);
\node [below right] at (1.5, 0) {1.5};
\end{tikzpicture}

}
	\caption{}
	\label{}
\end{marginfigure}

$L_{\infty}$

\begin{marginfigure}
	\centering
	\resizebox{7.5cm}{3cm}{%% page 16
%% fig 3
\begin{tikzpicture}
\draw[thick,-stealth] (-2, 0) -- (2, 0);
\draw[thick,-stealth] (0, -2) -- (0, 2);

\draw (-0.5, -0.5) rectangle (0.5, 0.5);
\node [below right] at (0.5, 0) {0.5};

\draw (-1, -1) rectangle (1, 1);
\node [below right] at (1, 0) {1};

\draw (-1.5, -1.5) rectangle (1.5, 1.5);
\node [below right] at (1.5, 0) {1.5};
\end{tikzpicture}
}
	\caption{}
	\label{}
\end{marginfigure}


Why might we be interested in different norms?


Ex: Later see applications in optimal control XXX want to meet a XXX objective (move a XXX from point a to XXX point b) while minimizing some resources $\rightarrow$ XXX will be XXX min a norm XXX XXX XXX 


$L_2$ $\rightarrow$ get min energy solution

$L_1$ $\rightarrow$ get a sparse solution not many forms of jet, useful when "XXX up" overhead

$L_\infty$ all uses of resource will be equal in XXX "XXX -XXX " XXX in XXX XXX

\vspace{0.5cm}
\noindent\textbf{Inner Products}

Final concept in geometry is angles which XXX to concept of inner product between elements of a vector space.

The "standard" inner product in $\reals_{n}$ (aka dot/scalar product)

$$x^{\trans}y=\sum_{k=1}^{n}x_{k}y_{k}$$

More general denote an inner product as $\langle x,y\rangle$

Definition: Any inner product on a (real) vector space $\Omega$ maps a pair of elements $x, y\in\Omega$ into the scalar, that is, $\langle \cdot,\cdot\rangle:\Omega\times\Omega\mapsto\reals$

For any $x, y, z\in\Omega$ and $a\in\reals$, the following holds:

$\langle x,y\rangle\geq 0$ and $\langle x,y\rangle=0$ iff $x=0\in\Omega$

$\langle x+y,z\rangle=\langle x,z\rangle+\langle y,z\rangle$

$\langle ax,y\rangle=a\langle x,y\rangle$

$\langle x,y\rangle=\langle y,x\rangle$


Note:
The above change slightly in complex vector space, e.g., $\langle x,y\rangle=\overline{\langle y,x\rangle}$

The concept we develop apply beyond list vectors in $\reals_{n}$ or $\complex_{n}$, e.g., space of polynomials or of XXX, but our focus will be $\reals_{n}$ and $\complex_{n}$.

Let's connect to angle now.

\begin{marginfigure}
	\centering
	\resizebox{7.5cm}{3cm}{%% page 18
\begin{tikzpicture}
\coordinate (O) at (0, 0);
\coordinate (A) at (4, 2);
\coordinate (B) at (1, 2);

\draw [ -stealth] (O) -- (A)node[below]{$x$};
\draw [ -stealth] (O) -- (B)node[above]{$y$};

\draw[dashed, -stealth] ($(O)!(B)!(A)$) -- (B)node[above, midway]{$\vec e$};

\draw [ -stealth] (O) -- ($(O)!(B)!(A)$)node[below right]{$x^{'}$};

\draw (A) -- (O) -- (B)
pic [draw= black, angle radius= 0.5cm, "$\theta$"] {angle = A--O--B};
\end{tikzpicture}
}
	\caption{}
	\label{}
\end{marginfigure}

Note: In above picture $x, y\in\reals_{n}$ but since $dim{span({x, y})}=2$ (assuming x and y are not co-linear). The familiar picture in $\reals_{2}$ shall holds.

Since we know that $\vert\cos\theta\vert <1$ XXX give

$$\vert\langle x,y\rangle\vert =\vert x^{\trans}y\vert \leq \Vert x\Vert_{2} \Vert y\Vert_{2}$$

Cauchy-Schwartz relates inner product(angle) to norms(length)

Holds for the inner products, not just in $\reals_{n}$

But can also related $\vert\langle x,y\rangle\vert$ to the norms (not $L_{2}$) via a generalization, "Holder's inequality"

$\vert x^{\trans}y\vert\leq \sum_{k=1}^{n}\vert x_{k}y_{k}\vert \leq \Vert x\Vert_{p}\Vert y\Vert_{q}$, for any $p, q\geq 1$ such that $1/p+1/q=1$

If $p=q=2$, get c-s

If $p=1$, $q=\infty$, get $\vert x^{\trans}y\vert\leq \Vert x\Vert_{1}\Vert y\Vert_{\infty}=(\sum_{k=1}^{n}\vert x_{k}\vert)(\max_{k\in [n]}x_{k})$



A second important connection of inner product and norm is that

$$\Vert x\Vert_{2}=\sqrt{x^{\trans} x}=\langle x,x\rangle$$

The $L_{2}$ norm is "induced" by the XXX inner product.

In fact, any inner product induces a norm (by the properties of inner product)

However, there are norms that are not induced by any inner product, e.g., $L_{1}$ and $L_{\infty}$.

Inner product space more special structure than a "normed" vector space.

\begin{marginfigure}
	\centering
	\resizebox{7.5cm}{3cm}{% page 19
\begin{tikzpicture}
    \path 
        (0, 0) rectangle (10, 6) [draw]
        (1, 5.5) node {vector space}

        (2.5,2) coordinate (A) node[below] { } ellipse (2 and 1.5) [draw]
        
        coordinate (temp) at (2.5, 2) ellipse (0.6 and 0.6) [draw]
        (temp) +(0.5, 0.75) coordinate (B) node[below left] { } -- (5, 4.5) -- (5.5, 4.5) node[right]{normed vector spaces}

        (A) -- (4.5, 3.5) -- (5, 3.5) node[right]{inner product spaces} ;

\end{tikzpicture}
}
	\caption{}
	\label{}
\end{marginfigure}

Note: There are also spaces with a sense of length (a "metric"). Those are not vector spaces (it can't add and scale elements). Those are "metric" spaces.

\vspace{0.5cm}
\noindent\textbf{Angles between vectors}

Important XXX: since by Cauchy-Schwartz $\frac{\vert \langle x,y\rangle\vert}{\Vert x\Vert \Vert y\Vert}\leq 1$

(a) If $\vert\cos\theta\vert = +1$, then $\theta 0^{\circ}$ or $180^{\circ}$. x and y are "co-linear", and $\vert\langle x,y\rangle\vert=\Vert x\Vert \Vert y\Vert$

\begin{marginfigure}
	\centering
	\resizebox{7.5cm}{3cm}{%% page 20
%% fig 1
\begin{tikzpicture}
\draw [white, <->] (0, 5) -- (0, 0) -- (5, 0);
\draw [-latex, ultra thick] (0, 0) -- (2, 2);
\node [below] at (3, 2) {$x$};
\node [below] at (2, 2) {$\cos e = 1$};
\draw [-latex, ultra thick] (2, 2) -- (4, 4);
\node [below] at (5, 4) {$y$};
\end{tikzpicture}
}
	\caption{}
	\label{}
\end{marginfigure}

\begin{marginfigure}
	\centering
	\resizebox{7.5cm}{3cm}{%% page 20
%% fig 2
\begin{tikzpicture}
\draw [white, <->] (0, 5) -- (0, 0) -- (5, 0);
\draw [-latex, ultra thick] (2, 2) -- (0, 0);
\node [below] at (1, 0) {$y$};
\filldraw[black] (2, 2) circle (2pt) node[anchor=west] {$\cos e = -1$};
\draw [-latex, ultra thick] (2, 2) -- (4, 4);
\node [above] at (5, 4) {$x$};
\end{tikzpicture}

}
	\caption{}
	\label{}
\end{marginfigure}

(b) Perhaps more important if $\vert\cos\theta\vert = 0$, then $\theta 90^{\circ}$, and $\frac{\vert\langle x,y\rangle\vert}{\Vert x\Vert \Vert y\Vert}=0$, or equivalently, $\langle x,y\langle=0$ asumming $x\neq 0$ and $y\neq 0$

\begin{marginfigure}
	\centering
	\resizebox{7.5cm}{3cm}{%% page 20
%% fig 3
\def\RightAngle[size=#1](#2,#3,#4){%
 \draw ($(#3)!#1!(#2)$) -- 
       ($($(#3)!#1!(#2)$)!#1!90:(#2)$) --
       ($(#3)!#1!(#4)$);
 \path (#3) --
 ($($(#3)!#1!(#2)$)!#1!90:(#2)$);
}
\begin{tikzpicture}
\coordinate (A) at (0, 2);
\coordinate (B) at (5, 2);
\coordinate (C) at (1, 0);

\draw [white, <->] (0, 5) -- (0, 0) -- (5, 0);

\draw [-latex, ultra thick] (C) -- (A);
\node [below] at (0, 3) {$x$};

\draw [-latex, ultra thick] (C) -- (B);
\node [below] at (4, 1) {$y$};

%% right angle
\RightAngle[size=10pt](B,C,A);

\end{tikzpicture}
}
	\caption{}
	\label{}
\end{marginfigure}

$\theta$ is a "right" angle, and x, y are orthogonal vectors.

(c) If $\vert \theta\vert <90^{circ}$ ,then $\cos\theta >0$. So $\langle x,y\rangle >0$ on "acute angle".

\begin{marginfigure}
	\centering
	\resizebox{7.5cm}{3cm}{%% page 20
%% fig 4
\begin{tikzpicture}
\draw [<-] (0, 5) -- (0, -1);
\draw [->] (-1, 0) -- (5, 0);

\draw [-latex, ultra thick] (0, 0) -- (5, 0);
\node [below] at (5, 0) {$x$};

\draw [-latex, ultra thick] (0, 0) -- (3, 3);
\node [below right] at (3, 3) {$y$};

\draw (1, 0) coordinate (A) -- (0, 0) coordinate (B) -- (1, 1) coordinate (C)
pic [draw= black, angle radius= 1cm] {angle};
\node [below right] at (1, 0.5) {$o$};
\end{tikzpicture}
}
	\caption{}
	\label{}
\end{marginfigure}

\begin{marginfigure}
	\centering
	\resizebox{7.5cm}{3cm}{%% page 20
%% fig 5
\begin{tikzpicture}
\draw [<-] (0, 5) -- (0, -1);
\draw [->] (-5, 0) -- (5, 0);

\draw [-latex, ultra thick] (0, 0) -- (5, 0);
\node [below] at (5, 0) {$x$};

\draw [-latex, ultra thick] (0, 0) -- (-3, 3);
\node [below right] at (-3, 3) {$y$};

\draw (1, 0) coordinate (A) -- (0, 0) coordinate (B) -- (-1, 1) coordinate (C)
pic [draw= black, angle radius= 1cm] {angle};
\node [below right] at (1, 1) {$o$};
\end{tikzpicture}}
	\caption{}
	\label{}
\end{marginfigure}

wheres if $\vert \theta\vert >90^{circ}$, so $\langle x,y\rangle <0$ on "abluse angle".

\vspace{0.5cm}
\noindent\textbf{Orthogonality}

A set of vectors $S=\{x^{(1)}, x^{(2)},\cdots, x^{(n)}\}$ is mutually orthogonal if $\langle x^{(i)},x^{(j)}\rangle =0$, $\forall i\neq j$
					
Such sets have nice property that the elements of S are linearly independent and so provide a basis for span(S) \& span(S)=m.

If, in addition, all elements have unit norm, i.e., $\Vert x^{i}\Vert_{2}=1$ for all $i\in[m]$ that the set forms an "orthogonal" basis.

Note that $\Vert\cdot\Vert_{2}$ XXX measure length XXX if is induced by the inner product.

When (shortly) get to projection will see orthogonal basis are easy to XXX XXX

\begin{marginfigure}
	\centering
	\resizebox{7.5cm}{3cm}{\input{./figures/ch02/p21-1}}
	\caption{}
	\label{}
\end{marginfigure}

\begin{marginfigure}
	\centering
	\resizebox{7.5cm}{3cm}{%% page 21
%% fig 2
\def\RightAngle[size=#1](#2,#3,#4){%
 \draw ($(#3)!#1!(#2)$) -- 
       ($($(#3)!#1!(#2)$)!#1!90:(#2)$) --
       ($(#3)!#1!(#4)$);
 \path (#3) --
 ($($(#3)!#1!(#2)$)!#1!90:(#2)$);
}
\begin{tikzpicture}
\node [below right] at (2, -1) {orthogonal};
\draw [<-] (0, 5) -- (0, -3);
\draw [->] (-3, 0) -- (5, 0);

\coordinate (A) at (3, -3);
\coordinate (B) at (-3, -3);
\coordinate (O) at (0, 0);

\draw [-latex, ultra thick] (O) -- (A);
\node [below] at (5, -5) {$x^{1}$};

\draw [-latex, ultra thick] (O) -- (B);
\node [below right] at (-3, -3) {$x^{2}$};

%% right angle
\RightAngle[size=10pt](B,O,A);

\end{tikzpicture}}
	\caption{}
	\label{}
\end{marginfigure}

Orthogonal complement: Given $S\in \gamma$, a subspace of $\gamma$, a vector $x\in\gamma$ is orthogonal to S if $x\perp s \forall s\in S$, i.e., $\perp$ to all vectors in S

$$ S^{\perp} =\{x\in\gamma|x\perp s\}$$

figure here

Some results:

(i) $S^{\perp}$ is a subspace: clearly include $0\in\gamma$ and is closed under linear combination(all linear combination $\perp S$)

(ii) dim($\gamma$)=dim(S) + dim ($S^{\perp}$)

(iii) Any $x\in\gamma$ can be written in a unique way as $x=x_{s}+x_{s^{perp}}$ for any subspace S

Note: If $S=\gamma$ then $S^{\perp}={0}$

\vspace{0.5cm}
\noindent\textbf{Projection}

Basic problem: Given a point $x\in\gamma$, find the "closeest" point in the set S (recall that points $\equiv$ vectors)

$$\Pi_{s}(x)=\arg \min \Vert y-x\Vert$$


(1) First for S a subspace of an inner product space, $L_{2}$

\begin{marginfigure}
	\centering
	\resizebox{7.5cm}{3cm}{%% page 23
%% fig 1
\begin{tikzpicture}[dot/.style={circle,inner sep=1pt,fill,label={#1},name=#1},
  extended line/.style={shorten >=-#1,shorten <=-#1},
  extended line/.default=1cm]
 \draw[thick,-stealth] (-2.5, 0) -- (4.5, 0);
 \draw[thick,-stealth] (0, -2.5) -- (0, 4.5);

 \coordinate (A) at (-1, -0.75);
 \coordinate (B) at (4, 3);
 \draw [extended line=0cm, -] (A) -- (B) node[pos=1.15, font=\small]{$\mathcal{S}$};     
 \draw [ -stealth] (0,0) -- (1.3, 2.15) coordinate (yn) node[right]{$x$};
 \draw[dashed] (yn) -- ($(A)!(yn)!(B)$);
 \draw [ -stealth] (0,0) -- ($(A)!(yn)!(B)$);
\end{tikzpicture}
}
	\caption{}
	\label{}
\end{marginfigure}

(2) Second for S, an "affine" set.

\begin{marginfigure}
	\centering
	\resizebox{7.5cm}{3cm}{%% page 23
%% fig 2
\begin{tikzpicture}[dot/.style={circle,inner sep=1pt,fill,label={#1},name=#1},
  extended line/.style={shorten >=-#1,shorten <=-#1},
  extended line/.default=1cm]
 \draw[thick,-stealth] (-2.5, 0) -- (4.5, 0);
 \draw[thick,-stealth] (0, -2.5) -- (0, 4.5);

 \coordinate (A) at (-1, 0.25);
 \coordinate (B) at (4, 4);
 \draw [extended line=0cm, -] (A) -- (B) node[pos=1.15, font=\small]{$\mathcal{A}$};     
 \draw [ -stealth] (0,0) -- (1, 3) coordinate (yn) node[right]{$x$};
 \draw[dashed] (yn) -- ($(A)!(yn)!(B)$);
\end{tikzpicture}}
	\caption{}
	\label{}
\end{marginfigure}

Basically a shift of a subspace

Work XXX of geometry

(3) Third will realize can do even for other norms, $L_{1}$, $L_{\infty}$ for which XXX inner product(projection in normed vectors space)


\noindent\textbf{Projection and 1-D subspace}

\begin{marginfigure}
	\centering
	\resizebox{7.5cm}{3cm}{%% page 24
\def\RightAngle[size=#1](#2,#3,#4){%
 \draw ($(#3)!#1!(#2)$) -- 
       ($($(#3)!#1!(#2)$)!#1!90:(#2)$) --
       ($(#3)!#1!(#4)$);
 \path (#3) --
 ($($(#3)!#1!(#2)$)!#1!90:(#2)$);
}
\begin{tikzpicture}
 \coordinate (A) at (-1, -0.75);
 \coordinate (B) at (4, 3);
 \coordinate (O) at (0, 0);

 \draw (A) -- (B) node[pos=1.15, font=\small]{$\mathcal{S}$};
 \filldraw[black] (O) circle (1pt) node[anchor=west] {$O$};
 \draw [ -stealth] (A) -- (3, 2.25) coordinate (yn) node[right]{$v$};
 \draw [ -stealth] (O) -- (1, 2.7) coordinate (yn) node[right]{$x$};
 \draw[dashed] (yn) -- ($(A)!(yn)!(B)$);
 \draw [ -stealth] (0,0) -- ($(A)!(yn)!(B)$) node[right]{$x_S$};
 
 \coordinate (S) at ($(A)!(yn)!(B)$);
 %% right angle
 \RightAngle[size=10pt](B,S,yn);

\end{tikzpicture}}
	\caption{}
	\label{}
\end{marginfigure}

$$S=span(\{v\})=\{\lambda v|\lambda \in \reals\}$$

Figure here

Be orthogonal decomposition  $x\in S\oplus S^{\perp}$

So $\exists x_{s}\in S, e\in S^{\perp}$

s.t. $x=x_{s}+e$, unique expression

$(x-x_{s})=e \in S^{\perp}$

Use this decomposition to solve optimization problem

$$\Pi_{s}(x)=\arg_{y\in S} \min \Vert y-x\Vert_{2}=\arg_{y\in S} \min \Vert y-x\Vert_{2}^{2}$$

Look at the objective function

\begin{align*}
\Vert y-x\Vert_{2}^{2}&=\langle y-x,y-x\rangle\\
&=\langle (y-x_{s})-e,(y-x_{s})-e\rangle\\
&=\Vert y-x_{s}\Vert^{2} + \Vert z\Vert^{2} - 2\langle y-x_{s},e\rangle\\
&\geq \Vert e\Vert_{2}^{2}
\end{align*}

where minimum attained by setting $y=x_{s}$. Note that the minimum is unique by uniqueness of orthogonal decomposition and $\Vert y-x_{s}\Vert^{2}=0$ iff $y=x_{s}$.



To summarize,

$$x_{s}=\Pi_{s}(x)=\arg_{y\in S} \min \Vert y-x\Vert_{2}$$

where $x_{s}$ is in $\perp$-decomposition.

To solve for $x_{s}$: use condition that

$x-x_{s} \perp S=\{\lambda v|\lambda\in\reals\}$

But $x\in S$, so $\exists a\in\reals$ s.t. $x_{s}=av$, need to solve for $a$.

$0=\langle x-av,v\rangle=\langle x,v\rangle-\langle av,v\rangle=\langle x,v\rangle-a\langle v,v\rangle$

so $a=\frac{\langle x,v\rangle}{a\langle v,v\rangle}=\frac{\langle x,v\rangle}{\Vert v\Vert^{2}}$

Thus, $x_{s}=av=\frac{\langle x,v\rangle}{\Vert v\Vert^{2}}v$

Notes: Recall earlier left XXX derivative that concept $\cos\theta$ to inner product

(1) $\cos\theta=\frac{x_{s}}{\Vert x\Vert}=\frac{1}{\Vert x\Vert}\frac{\vert\rangle x,v\rangle\vert}{\Vert v\Vert}\Vert v\Vert$

$\cos\theta = \frac{\vert\langle x,v\rangle\vert}{\Vert v\Vert\Vert x\Vert}$

(2) Nice way to remember

$x_{s}=\langle x,\frac{x}{\Vert v\Vert}\rangle \frac{x}{\Vert v\Vert}$

Now consider projection onto a subspace in general

figure here

Observe: all previous steps for 1-D case still hold. Only used S is 1-D when solving for $x^{(s)}$, so we have already done.

Theorem: Let $\Omega$ be an inner product space. Let $x\in\Omega$ and let $S\in\Omega$ be a subspace. There exists a unique vector $x^{*}\in S$

$x^{*} = \arg_{y\in S}\min \Vert x-y\Vert$

A necessary and sufficient condition for $x^{*}$ is

1. $x^{*} \in S$

2. $x-x^{*}\perp S$


\noindent\textbf{Solving for $x^{*}$}(general case)

Let $S=span\left(\{x^{(1)},x^{(2)},\cdots\,x^{(n)}\}\right)$

Since $x^{*}\in S$ can be written as 

$x^{*}=\sum_{i=1}^{d}a_{i}x^{(i)}$ for some(as yet unknown) $a_{i}$

since $(x-x^{*}) \perp S$

If $(x-x^{*}) \perp x^{(k)}$ $\forall k\in [d]$ that will be $\perp$ to all linear combination and hence $\perp to S$

Yields d conditions, $\forall k\in [d]$ we have

$0=\langle x-x^{*},x^{(k)}\rangle=\langle x-\sum_{i=1}^{d}a_{i}x^{(i)},x^{(k)}\rangle=\langle x,x^{(k)}\rangle-\sum_{i=1}^{d}a_{i}\langle x^{(i)},x^{(k)}\rangle$

Re-arranging yields

$\sum_{i=1}^{d}a_{i}\langle x^{(i)},x^{(k)}\rangle=\langle x,x^{(k)}\rangle \forall k\in [d]$

Or stacking into a matrix(d equations in d unknowns)
$$\left[ 
\begin{array}{cccc} 
\langle x^{(1)},x^{(1)}\rangle & \langle x^{(1)},x^{(2)}\rangle &\cdots& \langle x^{(1)},x^{(d)}\rangle\\
\vdots&&& \\
\langle x^{(d)},x^{(1)}\rangle & \cdots &\cdots& \langle x^{(d)},x^{(d)}\rangle
\end{array}
\right]
\left[ 
\begin{array}{c} 
d_{1}\\
\vdots\\
d_{d}
\end{array}
\right]=
\left[ 
\begin{array}{c} 
\langle x^{(1)},x\rangle\\
\vdots\\
\langle x^{(a)},x\rangle\\
\end{array}
\right]$$


One case where easy to solve the equations:

when the $x^{(k)}$ are all mutually $\perp$ matrix is diagonal.

If orthogonal and normalized matrix is identity matrix (even easier)

How do you orthogonal and normalize a matrix? Gran-Schmit Procedure

\begin{marginfigure}
	\centering
	\resizebox{7.5cm}{3cm}{%% page 28
%% fig 1
\begin{tikzpicture}
\coordinate (O) at (0, 0);
\coordinate (A) at (3, 0);
\coordinate (B) at (2, 2);

\draw [ -stealth] (O) -- (A) node[pos=1.15, font=\small]{$x^{(1)}$};
\draw [ -stealth] (O) -- (B) node[pos=1.15, font=\small]{$x^{(2)}$};
\end{tikzpicture}
}
	\caption{}
	\label{}
\end{marginfigure}

Step 1: Normalize $x^{(1)}$

$z^{(1)}=\frac{x^{(1)}}{\Vert x^{(1)}\Vert}$

\begin{marginfigure}
	\centering
	\resizebox{7.5cm}{3cm}{%% page 28
%% fig 2
\def\RightAngle[size=#1](#2,#3,#4){%
 \draw ($(#3)!#1!(#2)$) -- 
       ($($(#3)!#1!(#2)$)!#1!90:(#2)$) --
       ($(#3)!#1!(#4)$);
 \path (#3) --
 ($($(#3)!#1!(#2)$)!#1!90:(#2)$);
}
\begin{tikzpicture}
\coordinate (O) at (0, 0);
\coordinate (A) at (3, 0);
\coordinate (B) at (2, 2);

\draw [ -stealth] (O) -- (A) node[pos=1.15, font=\small]{$x^{(1)}$};
\draw [ -stealth] (O) -- (B) node[pos=1.15, font=\small]{$x^{(2)}$};

\coordinate (C) at ($(O)!(B)!(A)$);
\draw[dashed] (B) -- (C);

%% right angle
\RightAngle[size=10pt](A,C,B);

\draw [ -stealth] (O) -- (C) node[below]{$u$};
\draw [ -stealth] (O) -- (1, 0) node[below]{$z^{(1)}$};

\end{tikzpicture}
}
	\caption{}
	\label{}
\end{marginfigure}

Step 2: Orthogonal $x^{(2)}$

a.  Project $x^{(2)}$ and $z^{(1)}$

$\frac{\langle x^{(2)},z^{(1)}\rangle}{\Vert z^{(1)}\Vert} z^{(1)}=\langle x^{(2)},x^{(1)}\rangle z^{(1)}=u$


b. Normalize 

$\frac{x^{(2)}-u}{\Vert x^{(2)}-u\Vert}$

\begin{marginfigure}
	\centering
	\resizebox{7.5cm}{3cm}{%% page 28
%% fig 3
\begin{tikzpicture}
\coordinate (O) at (0, 0);
\coordinate (A) at (3, 0);
\coordinate (B) at (2, 2);
\coordinate (C) at (0, 3);

\draw [ -stealth] (O) -- (A) node[pos=1.15, font=\small]{$x^{(1)}$};
\draw [ -stealth] (O) -- (B) node[pos=1.15, font=\small]{$x^{(2)}$};

\draw[dashed] (B) -- ($(O)!(B)!(A)$);

\draw [ -stealth] (O) -- ($(O)!(B)!(A)$) node[below]{$u$};
\draw [ -stealth] (O) -- (1, 0) node[below]{$z^{(1)}$};

\draw (O) -- (C);
\draw[dashed] (B) -- ($(O)!(B)!(C)$);

\draw [ -stealth] (O) -- ($(O)!(B)!(C)$) node[left]{$x^{(2)}-u$};
\draw [ -stealth] (O) -- (0, 1) node[left]{$z^{(2)}$};
\end{tikzpicture}}
	\caption{}
	\label{}
\end{marginfigure}

XXX to higher dimensions as needed


Stacking up results to XXX yields "QR" decomposition

\begin{align*}
	A&=\left[\begin{matrix}
	\vdots&\vdots&\cdots&\vdots\\
	x^{1}&x^{2}&\cdots&x^{m}\\
	\vdots&\vdots&\cdots&\vdots
	\end{matrix}\right]=\left[\begin{matrix}
	\vdots&\vdots&\cdots&\vdots\\
	z^{1}&z^{2}&\cdots&z^{m}\\
	\vdots&\vdots&\cdots&\vdots
	\end{matrix}\right]
	\left[\begin{matrix}
	r_{11}&r_{12}&r_{13}&\cdots\\
	0&r_{22}&r_{23}&\cdots\\
	0&0&r_{33}&\cdots\\
	\vdots&\ddots& & 
	\end{matrix}\right]\\
	&=QR\\
	&=\left[\begin{matrix}
	\vdots&\vdots&\cdots\\
	r_{11}z^{1}&r_{12}z^{1}+r_{22}z^{1}&\cdots \\
	\vdots&\vdots&\cdots 
	\end{matrix}\right]
\end{align*}

To date what have seen in linear algebra class

Next something(perhaps) new: project onto affine set

\begin{marginfigure}
	\centering
	\resizebox{7.5cm}{3cm}{%% page 30
%% fig 1
\def\RightAngle[size=#1](#2,#3,#4){%
 \draw ($(#3)!#1!(#2)$) -- 
       ($($(#3)!#1!(#2)$)!#1!90:(#2)$) --
       ($(#3)!#1!(#4)$);
 \path (#3) --
 ($($(#3)!#1!(#2)$)!#1!90:(#2)$);
}
\begin{tikzpicture}[dot/.style={circle,inner sep=1pt,fill,label={#1},name=#1},
  extended line/.style={shorten >=-#1,shorten <=-#1},
  extended line/.default=1cm]
\draw[thick,-stealth] (-2.5, 0) -- (4.5, 0);
\draw[thick,-stealth] (0, -2.5) -- (0, 4.5);

\coordinate (O) at (0, 0) node[below right]{$O$};
\coordinate (A) at (4, 2);
\coordinate (B) at (3, 4);
\coordinate (D) at ($(O)!(B)!(A)$);

\draw [extended line=1.75cm, -] (O) -- (A) node[pos=1.5, font=\small]{$\mathcal{S}$};
\draw [ -stealth] (O) -- (B) node[right]{$x$};
\draw[dashed] (B) -- (D);
\draw [ -stealth] (O) -- (D) node[below]{$x_S$};

%% right angle
\RightAngle[size=10pt](B,D,O);

\end{tikzpicture}

}
	\caption{}
	\label{}
\end{marginfigure}

all subspace go through origin

can't be too difficult to project onto XXX line(subspace) that doesn't include origin

\begin{marginfigure}
	\centering
	\resizebox{7.5cm}{3cm}{%% page 30
%% fig 2
\def\RightAngle[size=#1](#2,#3,#4){%
 \draw ($(#3)!#1!(#2)$) -- 
       ($($(#3)!#1!(#2)$)!#1!90:(#2)$) --
       ($(#3)!#1!(#4)$);
 \path (#3) --
 ($($(#3)!#1!(#2)$)!#1!90:(#2)$);
}
\begin{tikzpicture}[dot/.style={circle,inner sep=1pt,fill,label={#1},name=#1},
  extended line/.style={shorten >=-#1,shorten <=-#1},
  extended line/.default=1cm]
\draw[thick,-stealth] (-2.5, 0) -- (4.5, 0);
\draw[thick,-stealth] (0, -2.5) -- (0, 4.5);

\coordinate (O) at (0, 0) node[below right]{$O$};
\coordinate (A) at (4, 3);
\coordinate (B) at (3, 4);
\coordinate (C) at (0, 1);
\coordinate (D) at ($(C)!(B)!(A)$);

\draw [extended line=3cm, -] (C) -- (A) node[below, pos=1.5, font=\small]{$\mathcal{A}$};
\draw [ -stealth] (O) -- (B);
\draw[dashed] (B) -- (D);

%% right angleS
\RightAngle[size=10pt](B,D,C);

\end{tikzpicture}}
	\caption{}
	\label{}
\end{marginfigure}

Seems there must be some way to XXX our results XXX to this slightly modified geometry.

\vspace{0.5cm}
Definition: an "affine" set is a shift/translate of a subspace

$A=\{x\in\Omega|x=u+x^{c}, u\in U, x^{(c)}\in A\}$

figure here

figure here

Note: can shift S be any point in A(since any other point in A is XXX point + a vector in S)

Idea of projection onto affine set

$\textcircled{0}$ To project $x\in \Omega$ onto A

$\textcircled{1}$ First translate both x and A by $-x^{(0)}$

translation of A is S

$\textcircled{2}$ Project(translate) $x-x^{(0)}$ onto S(as before)

shift result back by $+x^{(0)}$

$\textcircled{3}$ Get point in A

\begin{marginfigure}
	\centering
	\resizebox{7.5cm}{3cm}{%% page 32
%% fig 1
\begin{tikzpicture}
\draw[thick,-stealth] (-3, 0) -- (3, 0);
\draw[thick,-stealth] (0, -2) -- (0, 4);

\draw (-4, -0.5) -- (3, 3) node[right]{$\mathcal{A}$};    
\draw [dashed, -stealth] (0, 0) -- (1.6, 2.3)node[above]{$x^{(0)}$};
\draw [dashed, -stealth] (0, 0) -- (0.8, 3.4)node[above]{$x$};
\end{tikzpicture}
}
	\caption{}
	\label{}
\end{marginfigure}

\begin{marginfigure}
	\centering
	\resizebox{7.5cm}{3cm}{%% page 32
%% fig 2
\begin{tikzpicture}
\draw[thick,-stealth] (-3, 0) -- (3, 0);
\draw[thick,-stealth] (0, -2) -- (0, 4);

\draw (-3, -1.5) -- (3, 1.5) node[right]{$S=\mathcal{A}-x^{(0)}$};
\draw [dashed, -stealth] (-0.8, 1.1)node[left]{$x-x^{(0)}$} -- (0.8, 3.4)node[above]{$x$};
\end{tikzpicture}

}
	\caption{}
	\label{}
\end{marginfigure}

\begin{marginfigure}
	\centering
	\resizebox{7.5cm}{3cm}{%% page 32
%% fig 3
\def\RightAngle[size=#1](#2,#3,#4){%
 \draw ($(#3)!#1!(#2)$) -- 
       ($($(#3)!#1!(#2)$)!#1!90:(#2)$) --
       ($(#3)!#1!(#4)$);
 \path (#3) --
 ($($(#3)!#1!(#2)$)!#1!90:(#2)$);
}
\begin{tikzpicture}
\draw[thick,-stealth] (-3, 0) -- (3, 0);
\draw[thick,-stealth] (0, -2) -- (0, 4);

\coordinate (A) at (-3, -1.5);
\coordinate (B) at (3, 1.5);
\coordinate (C) at (-0.8, 1.1);
\coordinate (D) at ($(A)!(C)!(B)$);

\draw (A) -- (B) node[right]{$\mathcal{S}$};
\draw [dashed] (C)node[left]{$x-x^{(0)}$} -- (D)node[below]{$x_{\mathcal{S}}$};

%% right angle
\RightAngle[size=10pt](B,D,C);

\end{tikzpicture}
}
	\caption{}
	\label{}
\end{marginfigure}

\begin{marginfigure}
	\centering
	\resizebox{7.5cm}{3cm}{%% page 32
%% fig 4
\def\RightAngle[size=#1](#2,#3,#4){%
 \draw ($(#3)!#1!(#2)$) -- 
       ($($(#3)!#1!(#2)$)!#1!90:(#2)$) --
       ($(#3)!#1!(#4)$);
 \path (#3) --
 ($($(#3)!#1!(#2)$)!#1!90:(#2)$);
}
\begin{tikzpicture}
\draw[thick,-stealth] (-3, 0) -- (3, 0);
\draw[thick,-stealth] (0, -2) -- (0, 4);

\coordinate (A) at (-3, -1.5);
\coordinate (B) at (3, 1.5);
\coordinate (C) at (-0.8, 1.1);
\draw (A) -- (B) node[right]{$\mathcal{S}$};

\coordinate (A1) at (-4, -0.5);
\coordinate (B1) at (3, 3);
\coordinate (C1) at (0.8, 3.4);
\draw (A1) -- (B1) node[right]{$\mathcal{A}$};

\coordinate (D1) at ($(A1)!(C1)!(B1)$);
\draw[dashed] (C1)node[above]{$x$} -- (D1)node[below right]{$x^{*} = x_{\mathcal{S}} + x^{(0)}$};
\draw [dashed, -stealth] ($(A)!(C)!(B)$)node[below]{$x_{\mathcal{S}}$} -- (D1);

%% right angle
\RightAngle[size=10pt](B1,D1,C1);

\end{tikzpicture}}
	\caption{}
	\label{}
\end{marginfigure}


Theorem: Projection onto affine set

Let $\Omega$ be an inner product space

Let $A\in \Omega$ be an affine set where $A=S+x^{(c)}$

There is a unique $x^{*}\in A$ such that

$x^{*} = \arg_{y\in A}\min \Vert y-x\Vert$

A necessary and sufficient(set of) conditions:

1. $x^{*} \in A$

2. $(x-x^{*})\perp S$



Before XXX note $(x-x^{*})\perp S$ not that

If $(x-x^{*})\perp A$, then $(x-x^{*})\perp all vectors in A$

But can see not case

\begin{marginfigure}
	\centering
	\resizebox{7.5cm}{3cm}{%% page 33

\begin{tikzpicture}
\draw[thick,-stealth] (-3, 0) -- (3, 0);
\draw[thick,-stealth] (0, -2) -- (0, 4);

\coordinate (A1) at (-4, -0.5);
\coordinate (B1) at (3, 3);
\coordinate (C1) at (0.8, 3.4);
\coordinate (O) at (0, 0);

\draw (A1) -- (B1) node[right]{$\mathcal{A}$};

\node [above] at (C1) {$x$};
\draw[dashed, -stealth] ($(A1)!(C1)!(B1)$) -- (C1)node[above, midway]{$\vec e$};

\draw [ -stealth] (O) -- ($(A1)!(C1)!(B1)$)node[below right]{$x^{*}$};

\draw (C1) -- ($(A1)!(C1)!(B1)$) coordinate (D) -- (O)
pic [draw= black, angle radius= 0.5cm, "$e$"] {angle = C1--D--O};

\end{tikzpicture}}
	\caption{}
	\label{}
\end{marginfigure}

The book writes $(p-p^{(*)})\perp H$



Proof: any $y\in A$ can be expressed as $y=z+x^{(0)}$ when $z\in S$

\begin{align*}
\min_{y\in A} \Vert y-x\Vert&=\min_{(z+x^{(0)}\in) A} \Vert z+x^{(0)}-x\Vert\\
&=\min_{z\in S} \Vert z-(x-x^{(0)})\Vert
\end{align*}

Thus $z^{*}=\arg\min_{z\in S} \Vert z-(x-x^{(0)})\Vert$

and translating back,

$x^{(*)}=z^{(*)}+x^{(0)}$


What are the conditions for optimality?

$z^{(*)}-(x-x^{(0)})\perp S$ and $z^{*}$ by projection XXX

Thus, in terms of optimal $x^{*}$

$x^{(*)}=z^{(*)}+x^{(0)} \in A$

$z^{(*)}+x^{(0)}-x \perp S \equiv x^{(*)}-x\perp S$



Example: Projection onto a hyperplane

example here 

Exercise: Prove equivalence of 2 definition

Example: 2-D case




Back to projection onto hyperplane

$H=\{z\in\reals_{n}|a^{\trans} z=b\}=\{z\in\reals_{n}|z=x_{s}+z^{(0)},x_{s}\in S, z^{(0)}\in H\}$

Recall $dim(S)=n-1$, so $dim(S^{\perp})=1$

\begin{marginfigure}
	\centering
	\resizebox{7.5cm}{3cm}{%% page 39

\def\RightAngle[size=#1](#2,#3,#4){%
 \draw ($(#3)!#1!(#2)$) -- 
       ($($(#3)!#1!(#2)$)!#1!90:(#2)$) --
       ($(#3)!#1!(#4)$);
 \path (#3) --
 ($($(#3)!#1!(#2)$)!#1!90:(#2)$);
}

\begin{tikzpicture}
\draw[thick,-stealth] (-3, 0) -- (3, 0);
\draw[thick,-stealth] (0, -2) -- (0, 4);

\coordinate (A) at (-3, -1.5);
\coordinate (B) at (3, 1.5);
\coordinate (C) at (-0.8, 1.1);
\draw (A) -- (B) node[right]{$\mathcal{S}$};

\coordinate (A1) at (-4, -0.5);
\coordinate (B1) at (3, 3);
\coordinate (C1) at (0.8, 3.4);
\draw (A1) -- (B1) node[right]{$\mathcal{A}$};

\coordinate (D1) at ($(A1)!(C1)!(B1)$);
\draw[dashed] (C1)node[above]{$p$} -- ($(A1)!(C1)!(B1)$)node[below right]{$p^{*}$};
%% right angle
\RightAngle[size=10pt](C1,D1,A1);

\end{tikzpicture}}
	\caption{}
	\label{}
\end{marginfigure}

want $p^{(*)}=\arg \min_{p\in H} \Vert p^{*}-p\Vert$

Observe $p-p^{9*} \perp S$ (optimal condition)

So $p-p^{*} \in S^{(\perp)}=\{\lambda a|\lambda \in \reals \}$

Therefore, $\exists \lambda^{*}$ s.t.  $p-p^{*}=\lambda^{*}a$

want to solve for $\lambda^{*}$ but have 2 unknowns $(\lambda^{*},p^{*})$

will get rid of $p^{*}$ dependency by using definition of H

\begin{align*}
p-p^{*}&=\lambda^{*}a\\
a^{\trans} (p-p^{*})&=a^{\trans} (\lambda^{*}a)\\
a^{\trans} p-a^{\trans} p^{*})&=\lambda^{*}a^{\trans} a\\
a^{\trans} p-b&=\lambda^{*}a^{\trans} a\\
x^{*}=\frac{a^{\trans} p-b}{a^{\trans} a}\\
x^{*}=\frac{a^{\trans} p-b}{\Vert a\Vert^{2}}\\
\end{align*}

Thus, $p-p^{*}=\lambda^{*} a=(\frac{a^{\trans} p-b}{\Vert a\Vert^{2}})a$

$p^{*}=p-(\frac{a^{\trans} p-b}{\Vert a\Vert^{2}})a$

and 

$\Vert p-p^{*}\Vert=\Vert \lambda^{*}a\Vert= \vert \lambda^{*}\vert \Vert a\Vert=\frac{\vert a^{\trans} p-b\vert}{\Vert a\Vert}$

Recall terminology

$\Vert p-p^{*}\Vert=\min_{y\in H} \Vert y-p\Vert$

$p^{*}=\arg\min_{y\in H} p^{*}$


\vspace{0.5cm}
\noindent\textbf{Projection w.r.t other norms}

So far looked at projection in inner product space.

Recall inner product spaces have a notion of angle, have term "orthogonality principle", $L_{2}$ norm is one such example

In contrast, XXX $L_{1}$ and$ L_{\infty}$ norms don't come with a sense of angle. However the problem



still makes sens if $p\neq 2$, e.g., $p=1$, $p=\infty$, but cannot apply $\perp$ principle since no sense of angle

In following we will

1. discuss projection in normed vector spaces particularly $L_{1}$ and $L_{\infty}$

2. Illustrate how the solution differs as you change norm (change p)

3. Give you a sense for character of different such for $p=1$ and $p=\infty$

4. Get a sense of why might pick $p\neq 2$


Recall norm balls

\begin{marginfigure}
	\centering
	\resizebox{7.5cm}{3cm}{%% page 42
%% fig 1
\begin{tikzpicture}
\draw[thick,-stealth] (-2, 0) -- (2, 0);
\draw[thick,-stealth] (0, -2) -- (0, 2);

\draw (0,0) circle [radius=1];
\node [below right] at (1, 0) { };

\end{tikzpicture}
}
	\caption{}
	\label{}
\end{marginfigure}

\begin{marginfigure}
	\centering
	\resizebox{7.5cm}{3cm}{%% page 42
%% fig 2
\begin{tikzpicture}
\draw[thick,-stealth] (-2, 0) -- (2, 0);
\draw[thick,-stealth] (0, -2) -- (0, 2);

\draw (-1, 0) -- (0, 1) -- (1, 0) -- (0, -1) -- (-1, 0);
\node [below right] at (1, 0) { };
\end{tikzpicture}
}
	\caption{}
	\label{}
\end{marginfigure}

\begin{marginfigure}
	\centering
	\resizebox{7.5cm}{3cm}{%% page 42
%% fig 3
\begin{tikzpicture}
\draw[thick,-stealth] (-2, 0) -- (2, 0);
\draw[thick,-stealth] (0, -2) -- (0, 2);

\draw (-1, -1) rectangle (1, 1);
\node [below right] at (1, 0) { };

\end{tikzpicture}

}
	\caption{}
	\label{}
\end{marginfigure}

\begin{marginfigure}
	\centering
	\resizebox{7.5cm}{3cm}{%% page 42
%% fig 4
\begin{tikzpicture}
\draw[thick,-stealth] (-3, 0) -- (5, 0);
\draw[thick,-stealth] (0, -2) -- (0, 4);

\coordinate (O) at (0, 0) node[below right]{$O$};
\coordinate (A) at (-2, 2);
\coordinate (B) at (4.5, -1);
\draw (A) -- (B) node[right]{$\mathcal{A}$};
\end{tikzpicture}}
	\caption{}
	\label{}
\end{marginfigure}

Let's project $x=0\in\reals^{2}$ onto a line (affine set/hyperplane)

figure here

$$x^{*}=\arg\min_{x\in A} \Vert x-0\Vert_{p}=\arg\min_{x\in A} \Vert x\Vert_{p}$$


figure here

Observe:

$x_{2}^{*}$: Familiar with solution via inner product and $\perp$ theorem, closed form

$x_{1}^{*}$: Solution is "sparse", generally will be cost for other constraints since norm-ball XXX + XXX axis+aligned

$x_{\infty}^{*}$ at optimum, $x_{\infty, 1}^{*}=x_{\infty, 2}^{*}$, equal-magnitude coordinate

Applications(later in course in XXX x will be control vector over time)

$L_{2}$: energy control


$L_{2}^{*}$: sparse solution:


$L_{\infty}$:equal-effect: useful in "bang-bang" control, XXX XXX XXX XXX, e.g., in rockets


\vspace{0.5cm}
\noindent\textbf{Functions}

To date our focus has been on vectors

Now will discuss functions that map vectors inputs to real number. This is importance when come to optimization, e.g., in last example

$x^{*}=\arg\min_{y\in H} \Vert x-y\Vert$

$x\in\reals_{n}$ but the x is chosen to minimize the function $\Vert\cdot\Vert :\reals_{n}\mapsto\reals$

Some terminology in this book:

"Function":  $F:\reals_{n}\mapsto \reals$

"Map":  $F:\reals_{n}\mapsto \reals^{m}$

Note: above is like a XXX, XXX programming language
$$F: x \mapsto y$$

However, not all input values may be allowed, input may be a subset of $\Omega$ (cf, $\reals^{n}$), this is the "domain" of F

dom F= $\{x\in\reals^{n}|\vert F(x)\vert\leq\infty\}$



example here

figure here

figure here



not the same functions since domains differ.



Aside: Terminology when discussion a pair of vector space $(\gamma,u)$ over a field $\mathbb{F}$

F:$u\mapsto \gamma$, a "map", generally $dim(u)\neq dim(\gamma)$

F:$u\mapsto u$ , an "   ", input and output vectors spaces have the same dimension

F:$u\mapsto \mathbb{F}$ , a "function", map vector space into a scalar

in this course, $u=\reals_{n}$ , $\gamma=\reals_{m}$, $\mathbb{F}=\reals$(or, occasionally $\complex$)



Sets related to functions

Various sets defined by a function tell us a lot(or sometimes everything) about a function $F:\reals_{n}\mapsto\reals$

(1) The "graph" (a.k.a, "plot") of F is the set

graph F$=\{(x,F(x))\in\reals_{n+1}:x\in\reals_{n}\}$

(2) The "epigraph" of F is the set 

epf F$=\{(x,t)\in\reals_{n+1}:x\in\reals_{n}, t\geq F(x) \}$

figure here

figure here


also useful to consider points at(or below) a height

(3) The "level" set

$$c_{F}(t)=\{x\in\reals_{n}:F(x)=t\}$$

(4) The "Sub-level" set

$$L_{F}(t)=\{x\in\reals_{n}:F(x)\leq t\}$$

Note: graph and epigraph are in $\reals_{n+1}$, level and sublevel are in $\reals_{n}$


Let's sketch these sets for $L_{2}$ and $L_{1}$ norms in $\reals_{2}$

Graph:

figure here

Epigraph:

figure here

Level sets:

figure here

Sub-level sets:

figure here



Linear and affine functions: important classes

$F:\reals_{n}\mapsto\reals$ is linear iff

(1)"Homogeneous": $F(ax)=aF(x)$,$ \forall x\in\reals_{n}$ and $a\in\reals$

(2)"Additivity":  $F(x^{(1)}+x^{92})=F(x^{(1)})+F(x^{(2)})$

Put together and recurse to get 


$$F(\sum_{i\in [d]}^{}a_{i}x^{(i)})=\sum_{i\in [d]}^{}a_{i}F(x^{(i)})$$


$F:\reals_{n}\mapsto\reals$ is affine iff

$\overline{F}$ define pointwise as $\overline{F}=F(x)-F(0)$, $\forall x\in\reals_{n}$ is a linear function.


Turns out(wasn't prove-see "Linear Algebra Done right")

The $F:\reals_{n}\mapsto\reals$ is affine iff there is a unique pair $(a,b)\in\reals_{n}\times\reals$  s.t. 

$F(x)=a^{\trans} x+b$,  $\forall x\in\reals_{n}$

Since $F(0)=b$, this implies that any linear function can be expressed as $F(x)=a^{\trans} x=\langle a,x\rangle$ for some unique $a\in\reals_{n}$


Sets and linear/affine functions

The graph of $F:\reals_{n}\mapsto\reals$ is a 

+ subspace of $\reals_{n+1}$ if F is linear

+ hyperplane of $\reals_{n+1}$ if F is affine


The epigraph of $F:\reals_{n}\mapsto\reals$ is a 

+ half-space of $\reals_{n+1}$ if F is affine

+ half-space the boarder at which includes $0\in\reals_{n+1}$ if F is linear

figure here


Similar statements hold for level sets and sub-level sets in $\reals_{n}$, e.g., level sets of a linear function $F:\reals_{2}\mapsto\reals$ are affine sets in $\reals_{2}$

figure here


Exercise: Use definition of graph and epigraph to prove:

XXX round a best way to describe hyperplane and half-spaces, direct XXX next:

proof here


Proof: Graph F is a hyperplane when F is affine

proof here




Recalling definition of a hyperplane

$$H=\{z\in\reals_{n}|a^{\trans} z=b,a\in\reals_{n}, b\in\reals \}$$

Half-spaces are on one side or other of hyperplane

$$H_{+}=\{z\in\reals_{n}|a^{\trans} z>b\}$$

$$H_{-}=\{z\in\reals_{n}|a^{\trans} z\leq b\}$$

Recall that a is $\perp$ to the subspace but defines H, euqivalently the boarder of $H_{+}$ or $H_{-}$

figure here

Why is $H_{+}$(rather than $H_{-}$) the side of $H$ XXX which the normal direction is painting

figure here

Note a and $(z-z^{(c)})$ label the vectors, is the displacements, not the end-points. The end-points are labeled as $z$, $z^{(c)}$, $z^{(c)}+a$

Let's consider the inner product

\begin{align*}
\langle z-z^{c},(z^{c}+a)-a\rangle&=\langle z-z^{c},a\rangle\\
&=z^{\trans} a -(z^{(c)})^{\trans} a\\
&=z^{\trans}a-b
\end{align*}

\vspace{0.5cm}
To date

(1)Geometry

(2)Functions

In optimization you have some parametric vectors $x\in\reals_{n}$. You want to search all allowable x by moving around $\reals_{n}$(geometry!) to minimize some cost function $F:\reals_{n}\mapsto\reals$. Each parameter vector $x\in\reals_{n}$ will be associated with a cost $F(x)$.

How might you solve such problem?

Classic example in 2-D: Finding your way off a mountain and XXX of a faster to a town.

Anyone know the classic method? Follow a stream downhill.

Why is that a good method?

Eventually, would get to ocean, towns built by streams

In mountains no towns upstream.

But equally important, don't walk in circles since water doesn't flow uphill


To go downstream XXX easy to figure out when to get to stream - see which XXX water is flowing - Follow the negative gradient

In fact do some thing in $\reals_{n}$ not just $\reals_{2}$ "gradient descent".

So, gradient is importance, let's remind ourselves what it is.


Gradient

The gradient $\nabla F$ of $F:\reals_{n}\mapsto\reals$ is the vector of partial derivatives

$\nabla F= 
\left[ 
\begin{array}{c} 
\frac{\partial F(x)}{\partial x_{1}} \\
\frac{\partial F(x)}{\partial x_{2}} \\
\vdots \\
\frac{\partial F(x)}{\partial x_{n}}
\end{array}
\right]$,
where $x= 
\left[ 
\begin{array}{c} 
x_{1} \\
x_{2} \\
\vdots \\
x_{n}
\end{array}
\right]$

Sometime consider compound function, need chain rule for gradients

Say,

$g:\reals^{n}\mapsto\reals_{m}$

$F:\reals^{m}\mapsto\reals$

Both F and g are differentiable and we want $\nabla\Phi (x)$, where $\Phi (x)=F(g(x))$.

Before give formula, note that

1. $\nabla\Phi (x) \in\reals_{n}$ XXX they are n "lengths" XXX XXX in x.

2. Helps to consider an intermediate value z

3. First consider the k element of $\nabla\Phi$, i.e., $[\nabla\Phi(x)]_{k}$
\begin{align*}
[\nabla\Phi(x)]_{k}&=\frac{\partial\Phi (x)}{\partial x_{k}}\\
&=[\frac{\partial g_{1}(x)}{\partial x_{k}} \frac{\partial g_{2}(x)}{\partial x_{k}} \cdots \frac{\partial g_{m}(x)}{\partial x_{k}}] \nabla F(z)|_{z=g(x)}\\
&=\sum_{i=1}^{m} \frac{\partial g_{i}(x)}{\partial x_{k}} \frac{\partial F(z)}{\partial z_{i}}|_{z=g(x)}
\end{align*}

Stacking up 

$$
\nabla \phi (x)
=
\left[\begin{matrix}
	\frac{\partial \phi (x)}{\partial x_{1}}\\
	0\\
	0\\
	0\\
	\vdots\\
	\frac{\partial \phi (x)}{\partial x_{n}}\\
\end{matrix}\right]
=
\left[\begin{matrix}
	\frac{\partial g_{1}(x)}{\partial x_{1}}&\frac{\partial g_{2}(x)}{\partial x_{1}}&\cdots&\frac{\partial g_{m}(x)}{\partial x_{1}}\\
	\frac{\partial g_{1}(x)}{\partial x_{1}}& & &\vdots\\
	\vdots& & & \\
	\frac{\partial g_{1}(x)}{\partial x_{n}}&\cdots&\cdots&\frac{\partial g_{m}(x)}{\partial x_{n}}
\end{matrix}\right]
\nabla F(g(x))
$$

Example: $\Phi(x)=F(g(x))$, where $g:\reals_{n}\mapsto\reals_{m}$ is affine. 

In particular, $g(x)=\left[ 
\begin{array}{c} 
	g_{1}(x) \\
	g_{2}(x) \\
	\vdots \\
	g_{m}(x)
\end{array}
\right]$

and $g_{i}(x)=a_{i}^{\trans} x +b_{i}, i\in [m]$, $a_{i}\in\reals_{n}$, $b_{i}\in\reals$

\begin{align*}
\frac{\partial g_{i}(x)}{\partial x_{k}}=\frac{\partial}{\partial x_{k}}(a_{i}^{\trans} x+b)
&=\frac{\partial}{\partial x_{k}}(\sum_{j=1}^{m}a_{ij}x_{j})\\
&=a_{ik}
\end{align*}

Thus, $[\nabla\Phi(x)]_{k}=[a_{1k} a_{2k} \cdots a_{mk}]\nabla F(z)|_{z=g(x)}$

Stacking up we have

$$\nabla \phi (x)=
\left[\begin{matrix}
a_{11}&a_{21}&a_{31}&\cdots&a_{m1}\\
a_{12}&a_{22}&a_{32}&\cdots&a_{m2}\\
\vdots& & & & \\
a_{1n}&\cdots&\cdots&\cdots&a_{mn}\\
\end{matrix}\right]
\left[
\begin{matrix}
\frac{\partial F(z)}{\partial z_{1}}&\\
\vdots&\\
\vdots&\\
\frac{\partial F(z)}{\partial z_{n}}&
\end{matrix}\right]
$$



\textbf{Affine approximations}

Consider the Taylor series for $F\colon \mathbb{R}^n \to \mathbb{R}$.

$F(x) = F(x_0) + \nabla F(x_0)^{T} (x - x_0) + \varepsilon (x) $

$\lim_{x \to x_0} \frac{\varepsilon (x)}{\| x - x_0 \|_2} = 0$

Example

$F(x) = 2 x_1^2 + x_2^2$ XXX along $x_1$ axis XXX along $x_2$

$\nabla F(x) = \begin{bmatrix} 4x_1\\ 2x_2\\ \end{bmatrix}$

$F(x) \cong F(x^{(0)}) + \nabla F(x^{(0)})^{T} (x - x^{(0)})$
$\left. \nabla F(x) \right|_{x}$
(*)

\begin{displaymath}
\nabla F \left( \begin{bmatrix} 0\\ 0\\ \end{bmatrix} \right)  =
\begin{bmatrix} 0\\ 0\\ \end{bmatrix}
\end{displaymath}

\begin{displaymath}
\nabla F \left( \begin{bmatrix} 1\\ 0\\ \end{bmatrix} \right)  =
\begin{bmatrix} 4\\ 0\\ \end{bmatrix}
\end{displaymath}

\begin{displaymath}
\nabla F \left( \begin{bmatrix} 0\\ 1\\ \end{bmatrix} \right)  =
\begin{bmatrix} 0\\ 2\\ \end{bmatrix}
\end{displaymath}

figure here

$C_F(b) = {x \in \mathbb{R}^2 | F(x) = t}$


Sketch "level set" in 2-D

figure here

\begin{displaymath}
\nabla F \left( \begin{bmatrix} 1\\ 0\\ \end{bmatrix} \right)  =
\begin{bmatrix} 4\\ 0\\ \end{bmatrix}
\end{displaymath}

\begin{displaymath}
\nabla F \left( \begin{bmatrix} 0\\ -1\\ \end{bmatrix} \right)  =
\begin{bmatrix} 0\\ -2\\ \end{bmatrix}
\end{displaymath}

Let's visualize the set such that the increment in (*) is, 
the First order, constant.
I.e., which $x \in \mathbb{R}^2$ satisfy the relation

${x | \nabla F(x_0)^{T} (x-x_0) = c}$


Consider case $c=0$

There are points s.t., to approx (*), has some level as $F(x_0)$ when $c=0$

${x | \nabla F(x_0)^{T} (x-x_0) = 0}$

figure here


Consider $c = \varepsilon > 0$, a small positive XXX. Then, 
${x | \nabla F(x_0)^{T} (x-x_0) = \varepsilon}$
is points flat, to first order, have slightly higher cost (value, level). Then $F(x_0)$, XXX value at $x=x_0$.

figure here

In general the set
\begin{align*}
&\{x | \nabla F(x_0)^{T} (x-x_0) = c\}\\
&= \{x | \nabla F(x_0)^{T} x = \nabla F(x_0)^{T} x_0 + c\} \\
&= \{x | a^{T}x = b\}
\end{align*}
which is a "hyperplane", a type of affine set. 


Saw: - geometry of gradients connects to geometry of level sets. 

But you might think that is a bit funny. You know a Taylor sense approx is of the function $F$ not the level sets of $F$. 

You might also recall there is a tangent approximation involved somewhere, e.g. 

figure here

Taylor approx at $x = x_0$

To develop the approximation need to consider the plot or "graph" at the function $F$

$graph F = {(x, F(x) | x \in \mathbb{R}^n)} \subseteq \mathbb{R}^{n+1}$

eg in above example $F(x) = x^2+1$, $F \colon \mathbb{R} \to \mathbb{R}$, $s-n=1$ and plot (graph) is in $\mathbb{R}^2$

To find the tangent approximation, we will pick a point $t$ "above". The graph, is pick some pair $(x, t)$ s.t. $t \geq F(x)$

aside $t \in epi F$ in the "epigraph" of $F$

Use Taylor approximation above $x_0$ to approximate $F(x)$ 


figure here

Recap:

(1) Pick $(x, t)$ s.t. $t \geq F(x)$
 
(2) Assume $x$ and $x_0$ are "close" so approximation is accurate.

(3) By Taylor
$F(x) = F(x_0) + \nabla F(x_0)^{T} (x-x_0) + \varepsilon(x)$

(4) By (1), $t \geq F(x) = F(x_0) + \nabla F(x_0)^{T} (x-x_0) + \varepsilon(x)$

(5) By (2) will drop the $\varepsilon(x)$ XXX and assume inequality doesn't flip (because $x$ and $x_0$ are sufficiently close that $\varepsilon(x)$ is sufficient small)

Yields $t \geq F(x_0) + \nabla F(x_0)^{T} (x-x_0)$  (*)
Next, we re-arrange

(6)
\begin{align*}
0 &\geq -(t-F(x_0)) + \nabla F(x_0)^{T} (x - x_0) \\
&=\begin{bmatrix} \nabla F(x_0)^{T} -1\\ \end{bmatrix}  \begin{bmatrix} x-x_0\\ t-F(x_0)\\ \end{bmatrix} 
\end{align*}


Observe that:
$(x-x_{0})\in \reals^{n}$ "so vectors are in $\reals_{n+1}$"

$t-F(x_{0})\in\reals$ i.e., in example plot when $n=1$ in $\reals_{2}$.

figure here


Now recall connection between angles and inner products.
1. If inner product of 2 vectors is negative, then the angles is obtuse.
2. Matches picture.

What about vectors when this inner product=0?



$\{u\in\reals_{n+1}|<u, [\nabla F(x_{0}) , -1]^{trans}\}$

writing $u=[x-x_{0}, t-F(x_{0}]^{\trans}$ and no longer require $t\geq F(x)$.

The condition become
$$
\left[
\begin{array}{c} 
	x-x_{0}\\
	t-F(x_{0})
\end{array}
\right]^{\trans}  
\left[
\begin{array}{c} 
\nabla F(x_{0})\\
-1
\end{array}
\right]
=
0
$$

So we recognize the set defines a hyperplane in $\reals_{n+1}$

$H=\left\{\left[
\begin{array}{c} 
	x \\
    t
\end{array}
\right]  
\bigg|      
\left[
\begin{array}{c} 
x^{\trans}t
\end{array}
\right]  
\left[
\begin{array}{c} 
\nabla F(x_{0}) \\
-1
\end{array}
\right]  
=
\left[
\begin{array}{c} 
x_{0}^{\trans}F(x_{0})
\end{array}
\right]  
\left[
\begin{array}{c} 
	\nabla F(x_{0}) \\
	-1
\end{array}
\right]  
\right\}$

Finally, let's look at Taylor approximation one last time (1st order approximation)
\begin{align*}
F(x)&\approx + \nabla F(x_{0})^{\trans}(x-x_{0})\\
&=F(x_{0})+\Vert\nabla F(x_{0})\Vert \Vert x-x_{0}\Vert \left< \frac{\nabla F(x_{0})}{\Vert\nabla F(x_{0})\Vert},\frac{x-x_{0}}{\Vert x-x_{0}\Vert}\right>
\end{align*}
























 












\chapter{Matrices and eigen decomposition}
\label{ch.matEig}
\section{Matrices: array of numbers}

%Lecture 9.11


Matrices are rectangular arrays of numbers:

$$
A = 
\left[
\begin{matrix}
a_{11} & a_{12} & ... & a_{1n} \\
a_{21} & a_{22} & ... & a_{2n} \\
... & ... & ... & ...\\
a_{m1} & a_{m2} & ... & a_{mn}
\end{matrix}
\right] \in \Re^{m\times n}
$$


The element in the $i^{th}$ row \& $j^{th}$ column: $a_{ij} = [A]_{ij}$(equivalent notation)

The transposition operation works on matrices by exchanging rows and columns: 

$$
A^T = 
\left[
\begin{matrix}
a_{11} & a_{21} & ... & a_{1n} \\
a_{12} & a_{22} & ... & a_{m2} \\
... & ... & ... & ...\\
a_{1n} & a_{2n} & ... & a_{mn}
\end{matrix}
\right] \in \Re^{m\times n}
$$

So $[A]_{ij}  = [A^T]_{ji}$ if $A\in \Re^{m\times n}$ and $A^T\in \Re^{n\times m}$


1) $A + B = C$ where $[C] = [A]_{ij} + [B]_{ij}$

2) $\alpha A = B$ where $[B]l_{ij} = \alpha [A]_{ij}$

The origin is a all-zero matrix. 

\begin{definition}{Inner Product}
	$<A, B> = trace(A^TB) = trace(BA^T)$ where $A^TB\in \Re^{n\times n}$ and $B^TA\in \Re^{m\times m}$
\end{definition}

where trace(X) is defined as the sum of the diagonal elements of X. \\

Forbenius Norm: $||A||_F = \sqrt{<A, A>} = \sqrt{trace(A^TA)} = \sqrt{\sum^m_{i=1}\sum^m_{j=1}[A]^2_{ij}}$

$$\vec{(A)} = vec
\left(
\left[
\begin{matrix}
... & ... & ... & ... \\
a_{1} & a_{2} & ... & a_{n} \\
... & ... & ... & ...
\end{matrix}
\right]\right) = 
\left[
\begin{matrix}
a_{1} \\
a_{2} \\
... \\
a_{n}
\end{matrix}
\right]
\in \Re^{m\times n}
$$

Matrix inverse: if $A$ is "invertible", $\exists$ unique $A^{-1}$ s.t. $AA^{-1} = A^{-1}A = I$

\begin{itemize}
	\item $(AB)^{-1} = B^{-1}A^{-1}$
	
	\item $(A^{-1})^T = (A^T)^{-1}$
	
	\item $det(A^{-1}) = \frac{1}{det(A)}$
\end{itemize}

\subsection{Matrices as linear \& affine maps}


$x\in \Re^n \rightarrow A \rightarrow y = Ax \in \Re^m$\\

Affine maps are linear functions plus a constant term: $y = Ax + b$ where $A\in \Re^{m,n}$, $b\in \Re^m$

\subsection{Approximations}

A nonlinear map $f: \Re^n \rightarrow \Re^m$ can be approximated by an affine map:

\begin{equation*}
f(x) = f(x_0) + J_f(x_0)(x - x_0) + o(||x - x_0||)
\end{equation*}

where $o(||x - x_0||)$ are terms go to 0 faster than 1st order for $x\rightarrow x_0$ and $J_f(x_0)$ is the Jacobian of $f$ at $x_0$:




$$J_f(x_0) = 
\left[
\begin{matrix}
\frac{\sigma f_1}{\sigma x_1} &  ... & \frac{\sigma f_1}{\sigma x_n} \\
... &  ... & ...\\
\frac{\sigma f_m}{\sigma x_1} &  ... &\frac{\sigma f_m}{\sigma x_n}
\end{matrix}
\right]_{x = x_0}
$$

For $x$ 'near' $x_0$, the variation $\delta_f(x) =f(x) - f(x_0)$ can be approximated described by a linear map:

\begin{equation*}
\delta_f(x) = J_f(x_0)\delta_x, \,\, \delta_x = x - x_0
\end{equation*}
\subsection{Orthogonal Matrices}

\subsubsection{Orthogonal}

\begin{definition}
	$U\in \Re^{n\times m}$ is \textbf{orthogonal} if $U = [U^{(1)} ... U^{(m)}]$
	and 
	
	$$ U^{(1)^T}U^{(1)}=\left\{
	\begin{aligned}
	0\,\, \forall i\neq j \\
	1\,\, if \,\, i = j
	\end{aligned}
	\right.
	$$
\end{definition}

Then $UU^T = U^TU = I$\\


$x\rightarrow U \rightarrow y = Ux$

\begin{equation*}
||y||^2 = (Ux)^T(Ux) = x^TU^TUx = x^Tx = ||x||^2
\end{equation*}

$<Ux, Uw> = x^TU^TUw = x^Tw = <x, w>$

\subsubsection{Domain}

$dom(A)$ = $\Re^n$, $A = [a^{(1)}...a^{(n)}]$

$dom(A^T)$ = $\Re^m$, $A^T = [a^{(1)}...a^{(m)}]$


\subsubsection{Range}

Range of A is the set of vectors $y$ obtained as a linear combination of the $a_i$s are of the form $y= Ax$ for some vector $x\in \Re^n$.

\begin{equation*}
R(A) = \{y\in \Re^m | y = Ax = \sum^n_{i=1}x_ia^{(i)}\}
\end{equation*}

\begin{equation*}
R(A^T) = \{w\in \Re^m | w = A^Tu = \sum^m_{i=1}u_ia^{(i)}\}
\end{equation*}

\subsubsection{Rank}

The dimension of $R(A)$ is called the rank of $A$:

\begin{equation*}
rank(A) = dim\{R(A)\} = dim\{R(A^T)\} = rank(A^T)
\end{equation*}

\subsubsection{Nullspace} 

\begin{equation*}
N(A) = \{x\in \Re^n | Ax = 0\}
\end{equation*}

\subsubsection{Fundamental Theorem}
$\Re^n = \Re(A^T) \bigoplus N(A)$: $\forall x\in \Re^n$ there is a unique $x = x_{R(A^T)} = x_{N(A)}$ 

$N(A) \perp R(A^T)$, $N(A^T) \perp R(A)$


\section{PageRank}

%Lecture Sep 16

For the PageRank algorithm, more important webpages should be ranked higher:


Let us assume each node's importance score is evenly spread across all the outgoing links, each node’s importance score can also be written as the sum of the importance scores received from all of the incoming neighbors: for node A, $\sum_{j \rightarrow A}\frac{\pi_j}{O_j}$

%\begin{figure}
%	\centering
%	\resizebox{7.5cm}{3cm}{\input{./figures/ch03/figure0}}
%	\caption{}
%	\label{}
%\end{figure}


Let the score of node 1 (and 2) be x, and that of node 3 (and 4) be y. Looking at node 1’s incoming links, we see that there is only one such link, coming from node 4 that points to three nodes. So $x = \frac{y}{3}$ and $2x + 2y = 1$, so set of importance scores turns out to be [0.125, 0.125, 0.375, 0.375].

Matrix $H$: its (i,j)th entry is $\frac{1}{O_i}$ if there is a hyperlink from webpage i to webpage j, and 0 otherwise.

$\pi$: $N \times$1 column vector denoting the importance scores of the N webpages.

Multiply $\pi^T$ on the right by matrix $H$, this is spreading the importance score from the last iteration evenly among the outgoing links, and re-calculating the importance score of each webpage in this iteration by summing up the importance scores from the incoming links.

Note: $\pi^T[k] = \pi^T[k - 1]H$



Note: $\pi^T[k] = \pi^T[k - 1]G$

Obviously, $\pi^{T}$ is the left eigenvector of $G$ corresponding to the eigenvalue of 1: $\pi^{T} = \pi^{*T}G$.

For a graph like this:

\begin{figure}
	\centering
	\resizebox{7.5cm}{3cm}{\input{./figures/ch03/figure3.jpg}}
	\caption{}
	\label{}
\end{figure}


$$
\left[
\begin{matrix}
x_1 \\
x_2 \\
x_3\\
x_4\\
\end{matrix}
\right] =
\left[
\begin{matrix}
0 & 0 & 1 & \frac{1}{2} \\
\frac{1}{3} & 0 & 0& 0 \\
\frac{1}{3}& \frac{1}{2} & 0 & \frac{1}{2} \\
\frac{1}{3} & \frac{1}{2} & 0 & 0 
\end{matrix}
\right]
\left[
\begin{matrix}
x_1 \\
x_2 \\
x_3\\
x_4\\
\end{matrix}
\right]
$$

\begin{align*}
x &= Ax \\
Ax - x &= 0\\
(A - I)x &= 0\\
x&\in N(A-I)
\end{align*}


$$
x = \frac{1}{s_1}
\left[
\begin{matrix}
12 \\
4 \\
9\\
6\\
\end{matrix}
\right]
$$
Note: 

\begin{itemize}
	\item $x^{(0)}$ is the initial distribution.
	
	\item $x^{(0)}_i = $Pr[start on page $i$]
	
	\item $u^{(i)}, \lambda_i$ is eigen-vector/value pair if $Au^{(i)} = \lambda_iu^{(i)}$
\end{itemize}


If $Au^{(i)} = \lambda_iu^{(i)}$, $\bar{A}$ is "diagonalizable" if it has a full set of linear independent eigenvectors. In this case $x^{(0)} = \sum^n_{i=1}\alpha_iu^{(i)}$

\begin{align*}
x^{(1)} &= Ax^{(0)} = A[\sum^n_{i=1}\alpha_iu^{(1)} = \sum_i\alpha_i(Au^{(i)})]\\
x^{(2)} &= A(Ax^{(0)}) = \sum^n_{i=1}\alpha_i(A^2u^{(i)}) = \sum^n_{i=1}\alpha_i(\lambda_i^2u^{(i)})\\
...\\
x^{(k)} &= A^kx{(0)} = \sum^n_{i=1}\alpha_i(\lambda_i)^ku^{(i)}\\
&= \alpha_1(\lambda_1)^ku^{(1)} + \sum^n_{i=2}\alpha_i(\lambda_i)^ku^{(i)} \\
&= \alpha_1u^{(1)} + \sum^n_{i=2}\alpha_i(\lambda_i)^ku^{(i)}\\
&when \,k \rightarrow \infty\\
&= \alpha_1u^{(1)}
\end{align*}

\begin{equation*}
lim_{k\rightarrow \infty}\frac{A^kx^{(0)}}{||A^kx^{(0)}||} =u^{(i)}
\end{equation*}

%What is an Internet Structure s.t. $dim[N(A-I)] = 2$?

If have repeated eigenvalues:

\begin{align*}
Au^{(1)} =\lambda_1u^{(1)}\\
Au^{(2)} =\lambda_2u^{(2)}
\end{align*}

clearly:

\begin{equation*}
A(\alpha_1u^{(1)} + \alpha_2u^{(2)}) = \alpha_1\lambda_1u^{(1)} + \alpha_2\lambda_2u^{(2)}
\end{equation*}

1) The "algebraic" multiplicity of an eignevalue $\lambda$ of a square matrix $A$ is \# of eigenvalues $\lambda_i, \lambda_2,...,\lambda_m$ equal to $\lambda$. $\rightarrow$ AM($\lambda$)

2) The geometric multiplicity of an eigenvalue $\lambda$ of a square matrix $A$ is the dimension of $N(A - \lambda I)$. $\rightarrow$ GM($\lambda$)

In general, $0 < GM(\lambda) \leq AM(\lambda)$ \& If $GM(\lambda_i) = AM(\lambda_i)$, $\forall i$, then $A$ is diagonalizable. 


If diagonalizable, we can write $Au^{(i)} = \lambda_iu^{(i)} \forall$ assume all $\lambda_i$ distinct $GM(\lambda_i) = AM(\lambda_i) = 1, \forall i$

$$
\left[
\begin{matrix}
Au^{(1)} & Au^{(2)} &... &Au^{(n)} 
\end{matrix}
\right] =
\left[
\begin{matrix}
\lambda_1u^{(1)} & \lambda_2u^{(2)}&... &\lambda_iu^{(i)}
\end{matrix}
\right]
$$

$$A
\left[
\begin{matrix}
u^{(1)} & u^{(2)} &... &u^{(n)} 
\end{matrix}
\right] =
\left[
\begin{matrix}
u^{(1)} & u^{(2)} &... &u^{(n)}
\end{matrix}
\right]
\left[
\begin{matrix}
\lambda_1 & 0 & ... & 0\\
0& \lambda_2  &  ... & 0\\
...  & ...  &   ...& \\
0    &  ... &  0 & \lambda_n
\end{matrix}
\right]
$$



\begin{align*}
AU &= U\Lambda\\
A &= U\Lambda U^{-1}\\
\Lambda &= U^{-1}AU
\end{align*}

Recall pagerank:

\begin{align*}
A^kx^{(0)} &= (U\Lambda U^{-1})^kx^{(0)}\\
&=U\Lambda^kU^{-1}x^{(0)} \\
&= U
\begin{bmatrix}
\lambda_1^k & 0 & ... & 0\\
0& \lambda_2^k  &  ... & 0\\
...  & ...  &   ...& \\
0    &  ... &  0 & \lambda_n^k
\end{bmatrix} U^{-1}x^{(0)}
\end{align*}

\subsection{Determinant}

In eigen-decomposition, we need to solve for $\lambda$ from $det(A - \lambda I) = 0$. 
If $det(A) = 0$ then $A$ is non-invertible. 

Example:

$$A = 
\left[
\begin{matrix}
a_{11} & a_{12}\\
a_{21} & a_{22}
\end{matrix}
\right] =
\left[
\begin{matrix}
a_{(1)} & a_{(2)}
\end{matrix}
\right]
$$

\begin{align*}
U &= \{x\in \Re^2 | 0\leq x_1 \leq 1, 0\leq x_1 \leq 1 \}\\
P &= \{Ax \ x\in \mathcal{U}\} 
\end{align*}


%\begin{figure}
%	\centering
%	\resizebox{7.5cm}{3cm}{\input{./figures/ch03/figure1}}
%	\caption{}
%	\label{}
%\end{figure}

Assume $A$ is diagonalizable:

\begin{align*}
A &= U\Lambda U^{-1}\\
|det(A)| &= |det(U\Lambda U^{-1})| \\
&= |det(U)det(\Lambda)det(U^{-1})|\\
&= det(U)det(\Lambda)\frac{1}{det(U)}\\
&= |det(\Lambda)|\\
&= |\prod^n_{i=1}\lambda_i|
\end{align*}

\begin{align*}
X: N(\mu, \Sigma)where(\mu \in \Re^n, \Sigma \in \Re^{n\times n})\\
P_x(x) = \frac{1}{(2\pi)^{\frac{2}{n}}}\frac{1}{|det\Sigma|^{\frac{1}{2}}}exp[-\frac{1}{2}(x - \mu)^T\Sigma^{-1}(x - \mu)]
\end{align*}




\chapter{Symmetric matrices and spectral decomposition}
\label{ch.symmMat}
%% Placeholder for chapter on symmetric matrices and spectral decomposition


\chapter{Singlar value decomposition}
\label{ch.SVD}
%% Placeholder for chapter on SVD


%\chapter{Linear equations and least squares}
%\label{ch.linEqLS}
%%% Placeholder for chapter on linear equations and least squares



last time

solving linear system of equations Ax=y

1. over determined

$A^TAx=A^Ty$


2. under determined m<n

$X^* \in R(A^T)  $ 

3. invertiable





Today

interpretation and variations 


$min_x \Vert y-Ax\Vert_2$

1 approximate solution to y=Ax

$y^*=Ax^*$ best approx to solution in l2 space

closes point in R(A) to y

2. minimum perturbation of y to feasibility"

$y+\Delta y= y^*$

3. perturb both y and A to get feasibility

total least square

$\min_{\Delta y \Delta A} \Vert [\Delta A \Delta y] \Vert_F$, $\Delta A$ is m by n matrix and so $[\Delta A \Delta y]$ is m by n+1

% $\min_{\Delta y \Delta A} \Vert [\Delta A \Delta y] \Vert_F =  \lambda \Vert [\Delta A] \Vert_F + \Vert [\Delta y] \Vert_F $


$y+\Delta y\in R(A+\Delta A)$


4. Linear regression

$\Vert y-Ax\Vert ^2_2 = \sum_{i=1}^{m} ( y_i - <a^{(i)} , x>)^2$


examples

fit a line to $\{(0,6),(1,0),(2,0) \}={(a_i,y_i)}$

approximation of form $x_1+ax$

want $(y_i-x1+a_ix_i)^2 = r_i^2$

$x^*=(A^TA)^{-1}A^Ty=[5 , -3]^T$


$\hat{y}=x1^*+ax2^*=5-3a$            %so x1 and x2 are parameters


variants of ls

some residuals are more important than the others.

$\min \sum_{i=1}^{n} w_i r_i=\Vert W (y-Ax)\Vert ^2_2=\Vert Wy-WAx)\Vert ^2_2=\Vert \bar{y}-\bar{A} \Vert ^2_2$

can use a more general transform 

$\Vert W (y-Ax)\Vert ^2_2 = (y-Ax)^TW^TW(y-Ax)=r^TW^TWr$

$x^*=(A^TW^TWA)^{-1}A^TWW^Ty$




weighted least square (ls) 

when W is diagonal 

figure here

when W is PSD (rotation)

figure here




Regularization LS

l2 regularized LS

in ls $x*=\arg_x\in\reals_{n} \min \Vert Y-Ax\Vert ^2_2$

and no preference for any specific x over any other, and often x is a vector of resources consumed.

regularized ls

$x*=\arg_x\in\reals_{n} \min \Vert Y-Ax\Vert ^2_2 + \gamma \Vert x\Vert ^2_2$, where $\gamma$ is a non negative scalar


To solve regularized ls

$u\in\reals_{n}, v\in\reals_{m}$


Define

%\bar{A}=

$\Vert Y-Ax\Vert ^2_2 + \gamma \Vert x\Vert ^2_2=\Vert \bar{A}x-\bar{y}\Vert^2_2$

$x^*=(\bar{A}^TA)^{-1}\bar{A}^T\bar{y}=(A^TA+\gamma I)^{-1}A^Ty$


tiknon regularization

content here


Visualize regularized LS

$\min \Vert Ax-y\Vert_2^2 +\gamma\Vert x\Vert^2_2$

recall $x^*=[5 -3]^T$, and $\Vert Ax^*-y\Vert = 6$

figure here


Draw level set for some picture

figure here

figure here



$c_1=\Vert Ax-y\Vert^2_2=\Vert A(x-x_s^*+x_s^*)-y\Vert^2_2=\Vert (Ax_s^*-y)-A(x-x_s^*)\Vert^2_2=\Vert (Ax_s^*-y)\Vert^2_2 - \Vert A(x-x_s^*)\Vert^2_2$

The first term on r.h.s is a scalar 6, so we focus on the second term now.

$\Vert A(x-x_s^*)\Vert^2_2= (x-x_s^*)^T A^TA (x-x_s^*)$ 

Note that $A^TA$ is a PSD matrix.


Understand geometry level set of $\Vert A(x-x_s)\Vert^2_2$ via eigenvector of PSD matrix $A^TA$


$A^TA=$

figure here





















%\chapter{Linear, quadratic, and geometric models}
%\label{ch.linQuadGeom}
%%% Placeholder for chapter on linear, quadratic, and geometric models

\section{Linear Programs: An Optimization Problem}

Also known as(a.k.a) (mathematical program") of the form

\begin{align*}
(arg)\,\,min_{x\in \Re^n}&c^Tx + d\,\,\,\text{("objective"function)}\\
s.t.\,\,\, &Ax = b\\
&Gx \leq h
\end{align*}

"Feasible" set(might not be good idea, but you can do this): $S = \{x|Ax = b, Gx \leq h \}$


\begin{itemize}
	\item $c\in \Re^n$, $d\in \Re$.
	
	\item $A\in \Re^{q\times n}$, $b\in \Re^q$, \begin{equation*}
	A = 
	\begin{bmatrix}
	\alpha^{(1)^T}\\
	...\\
	\alpha^{(q)^T}
	\end{bmatrix}
	\end{equation*}
	
	$<\alpha^{(i)}, x> =b_i$, $i\in \left[q\right]$
	
	\item $G\in \Re^{m\times n}$, $h\in \Re^m$
	
	\item If
	\begin{equation*}
	G = 
	\begin{bmatrix}
	g^{(1)^T}\\
	...\\
	g^{(q)^T}
	\end{bmatrix}
	\end{equation*}
	
	Then $<G^{(i)}, x>\leq h_i$, $i\in \left[m\right]$
	
\end{itemize}





\begin{align*}
p^* = min \,\,\, &c^Tx + d\\
s.t.\,\,\, &Ax = b\\
&Gx\leq h
\end{align*}

"optimal" value of program:

\begin{itemize}
	\item Lowest cost shoice amongst all feasible $x$.
	
	\item Possible here is no minimal choice
	
	\item possible no feasible choice
	
	\item $p^*\in \Re$
\end{itemize}


\begin{align*}
x^* = min \,\,\, &c^Tx + d\\
s.t.\,\,\, &Ax = b\\
&Gx\leq h
\end{align*}

$x^*$ "optimal" choice of optimization variable(or vector):

\begin{itemize}
	\item sometimes $x^*$ does not exist
	
	\item if exists, may or may not be unique
	
	\item $x^*\in \Re^n$
\end{itemize}


Let's consider an example:\\

During the The Second World War, the US army is considering how to make their soldiers have enough nutrients...\\

Different nutrients in different foods and daily requirement:\\


\begin{tabular}{|c|c|c|c|}
	\hline 
	Nutrients&Meat&Potatoes&Daily Requirement\\
	\hline  
	Carbohydates&40&200&400\\
	\hline  
	Protein&100&20&200\\
	\hline  
	Fiber&5&40&40\\
	\hline 
\end{tabular}\\

The price of meat and potatoes:\\



\begin{tabular}{|c|c|}
	\hline 
	Resources&cost/kg\\
	\hline  
	Meat &\$ 1\\
	\hline 
	Potatoes &\$ 0.25\\
	\hline 
\end{tabular}

$x_1$ denotes meat(kg) and $x_2$ denotes potatoes(kg).\\


Objective:

\begin{equation*}
x_1 + \frac{1}{4}x_2 = 
\begin{bmatrix}
1 & \frac{1}{4}
\end{bmatrix}
\begin{bmatrix}
x_1\\
x_2
\end{bmatrix}
\end{equation*}

Constrains:
\begin{align*}
40x_1 + 200x_2 &\geq 400\\
100x_1 + 20x_2 &\geq 200\\
5x_1 + 40x_2 &\geq 40\\
x_1 \geq 0\\
x_2 \geq 0
\end{align*}

$Gx\leq h \rightarrow$ 

\begin{equation*}
\begin{bmatrix}
-\frac{1}{5} & -1\\
-\frac{1}{8} & -1\\
-5 & -1\\
-1 & 0\\
0 & -1
\end{bmatrix}
\begin{bmatrix}
x_1\\
x_2
\end{bmatrix}\leq
\begin{bmatrix}
-2\\
-1\\
-10\\
0\\
0
\end{bmatrix}
\end{equation*}

\begin{figure}
	\centering
	\includegraphics[width=2.1in,height=2.1in]{figures/ch06/figure5.png}
	%\caption{This is an inserted JPG graphic} 
	%\label{fig:graph} 
\end{figure}

%Below are notes for Oct 12
We have started talking about Linear Programming problems:

\begin{align*}
(arg)min_{x\in \Re^n} \,\,\, c^Tx + d\\
s.t. \,\,\, Ax& = b\\
Gx &\leq h\\
A&\in \Re^{q\times n}, b\in \Re^{q}\\
G&\in \Re^{m\times n}, h\in \Re^{m}
\end{align*}

Objective of LP, have or don't have constraints

\begin{align*}
p^* &= min \,\,\, c^Tx+d\\
x^* &= arg\,\, min_{x\in \Re^n} \,\,\, c^Tx+d
\end{align*}

Probability 1: $c = 0 \in \Re^n$

\begin{align*}
p^* &= min_{x\in \Re^n} \,\,\, d=d\\
x^* &= arg\,\, min_{x\in \Re^n} \,\,\, d = \Re^n
\end{align*}

Probability 2: $c \neq 0 \in \Re^n$

\begin{align*}
p^* &= -\infty\,\,\, \text{by convention if no minimum}\\
x(\alpha) &= -\alpha c \,\,\, \alpha \geq 0\\
c^Tx + d &= c^T(-\alpha c) + d = \alpha - \alpha c^Tc = \alpha - \alpha||c||^2_2\\
x^* &\text{doesn't exist}
\end{align*}

So for unconstrained LP:

\begin{equation*}
%\label{eq6}
p^*=\left\{
\begin{aligned}
d &  & \text{if } c=0 \\
-\infty &  & \text{otherwise}
\end{aligned}
\right.
\end{equation*}

\begin{equation*}
%\label{eq6}
x^*=\left\{
\begin{aligned}
\Re^n & &\text{if } c=0 \\
\text{doesn't exist} &  & \text{otherwise}
\end{aligned}
\right.
\end{equation*}

Think about geometry of cost function:

\begin{equation*}
F_0(x)(\text{"objective function"}) =c^Tx + d(affine function)
\end{equation*}

\begin{align*}
c_{F_0}(t) &= x\in \Re^n | F_0(x) = c^Tx + d = t \}(\text{hyper-plane})\\
&= \{x\in \Re^n | C^Tx = (t-d) \}(\text{offset when t=d}, C_{F_0}(t) \text{is a subspace})
\end{align*}

\begin{figure}
	\centering
	\includegraphics[width=2.1in,height=2.1in]{figures/ch07/figure1012_1.png}
	%\caption{This is an inserted JPG graphic} 
	%\label{fig:graph} 
\end{figure}

To find the relationship between $t_1$ and $t_2$:

\begin{align*}
t_2 &= c^T(x^{(0)}+\aleph c) + d\\
t_1 &= c^Tx^{(0)} + d\\
t_2 - t_1 &= [c^Tx^{(0)} + d + \alpha ||c||^2] - c^Tx^{(0)} = \alpha \Vert c \Vert^2
\end{align*}

One approach:

%\begin{align*}
%t_2 &= c^T(x^{(0)} + \alpha c) + d\\
%t_1 &= c^Tx^{(0)} + d\\
%t_2 - t_1 &= \[c^Tx^{(0)} + d + \alpha||c||^2 \] - c^Tx^{(0)} + d\\
%&= \alpha ||c||^2\\
%\end{align*}

\begin{equation*}
\nabla F_0(x) =
	\begin{bmatrix}
	\frac{\sigma}{\sigma x_1}(c^Tx+d)\\
	\vdots\\
	\frac{\sigma}{\sigma x_n}(c^Tx+d)
	\end{bmatrix} = 
	\begin{bmatrix}
	c_1\\
	c_2\\
	\vdots\\
	c_n
	\end{bmatrix} = c
\end{equation*}

\begin{align*}
Ax &= b\\
Gx &\leq h
\end{align*}

Then it comes to:

\begin{equation*}
A = 
\begin{bmatrix}
\alpha^{(1)}\\
\alpha^{(2)}\\
\vdots\\
\alpha^{(q)}
\end{bmatrix}
\end{equation*}


\begin{equation*}
\{x|Ax = b \} =  \cap^q_{i=1}\{x|<\alpha^{(i)}, x> = b_i \}
\end{equation*}



The "feasible" set:

\begin{equation*}
S = \left(\cap^q_{i=1}\{x\in \Re^n | <\alpha^{(i)}, x> = b_i \}\right) \cap \left(\cap^m_{i=1}\{x\in \Re^n | <g^{(i)}, x> \leq h_i \}\right)
\end{equation*}

(Intersect half-spaces with hyper-planes)

\begin{itemize}
	\item "polyhedron" / "polytape"
	
	\item Ax = b $\rightarrow$ $Ax \leq b$, $Ax \geq b$
\end{itemize}

\begin{example}
	\begin{align*}A &= 
	\begin{bmatrix}
	1 & 1
	\end{bmatrix}
	b = 
	\begin{bmatrix}
	2
	\end{bmatrix}\\
	G &= 
	\begin{bmatrix}
	-1 & 0\\
	0 & -1
	\end{bmatrix}
	h = 
	\begin{bmatrix}
	0\\
	0
	\end{bmatrix}\\
	Ax &= b \rightarrow x_1 + x_2 = 2\\
	Gx &\leq h \rightarrow x_1 \geq 0, x_2\geq 0
	\end{align*}
	
	
	\begin{figure}
	\centering
	\includegraphics[width=2.1in,height=2.1in]{figures/ch07/figure1012_2.png}
	%\caption{This is an inserted JPG graphic} 
	%\label{fig:graph} 
	\end{figure}
\end{example}








\begin{example}
	Equality constraints force you into lower effective dimension:
	
	\begin{align*}
	A =
	\begin{bmatrix}
	1&1&1
	\end{bmatrix}
	\end{align*}
	\begin{equation*}
	B= 
	\begin{bmatrix}
	1\\
	\end{bmatrix}
	\end{equation*}
	
	\begin{figure}
	\centering
	\includegraphics[width=2.1in,height=2.1in]{figures/ch07/figure1012_3.png}
	%\caption{This is an inserted JPG graphic} 
	%\label{fig:graph} 
	\end{figure}
\end{example}

\begin{example}
	\begin{equation*}
	\begin{bmatrix}
	1&1\\
	1&1
	\end{bmatrix}
	\begin{bmatrix}
	x_1\\
	x_2
	\end{bmatrix}=
	\begin{bmatrix}
	1\\
	2
	\end{bmatrix}
	\end{equation*}
	
	\begin{figure}
	\centering
	\includegraphics[width=2.1in,height=2.1in]{figures/ch07/figure1012_4.png}
	%\caption{This is an inserted JPG graphic} 
	%\label{fig:graph} 
	\end{figure}
	
	Can also happen with inequalities constraints:
		\begin{align*}
	\begin{bmatrix}
	-1&0\\
	1&0
	\end{bmatrix}
	\begin{bmatrix}
	x_1\\
	x_2
	\end{bmatrix}&\leq
	\begin{bmatrix}
	0\\
	-1
	\end{bmatrix}\\
	S &= \emptyset
	\end{align*}

\end{example}



\begin{example}
	\begin{align*}
	\begin{bmatrix}
	-1 & 0\\
	0 & -1\\
	\frac{1}{2} & -1\\
	1 & 1
	\end{bmatrix}
	\begin{bmatrix}
	x_1\\
	x_2
	\end{bmatrix}
	 &\leq 
	 \begin{bmatrix}
	 0\\
	 0\\
	 1\\
	 3
	 \end{bmatrix}\\
	 \frac{1}{2}x_1 - x_2&\leq 1\\
	 x_2 &\geq -1 + \frac{1}{2}x_1
	\end{align*}
	
	\begin{figure}
	\centering
	\includegraphics[width=2.1in,height=2.1in]{figures/ch07/figure1012_5.png}
	%\caption{This is an inserted JPG graphic} 
	%\label{fig:graph} 
	\end{figure}
\end{example}





LP: Linear objective \& constraints

\begin{figure}
	\centering
	\includegraphics[width=2.1in,height=2.1in]{figures/ch07/figure1012_6.png}
	%\caption{This is an inserted JPG graphic} 
	%\label{fig:graph} 
\end{figure}

\begin{equation*}
S = \{x\in \Re^3 | 
\begin{bmatrix}
1 & 1 & 1
\end{bmatrix}
\begin{bmatrix}
x_1\\
x_2\\
x_3
\end{bmatrix}
 = 1, x_1 \geq 0, x_2\geq 0,x_3\geq 0
 \}
\end{equation*}

\begin{align*}
x^* = arg\,\,\, &min 
\begin{bmatrix}
1&1&1
\end{bmatrix}x\\
&s.t. \,\,\, x\in S\\
x^* = arg\,\,\, &min 
\begin{bmatrix}
1&0&0
\end{bmatrix}x = x_1\\
\end{align*}

\begin{figure}
	\centering
	\includegraphics[width=2.1in,height=2.1in]{figures/ch07/figure1012_7.png}
	%\caption{This is an inserted JPG graphic} 
	%\label{fig:graph} 
\end{figure}

So there are various possibilities here for $p^* \& x^* $:

\begin{itemize}
	\item 1) $x^*$ is unique, $p^*$ finite
	
	\item 2) $x^*$ is not unique, $p^*$ finite. 
	
	\item 3) There is no $x^*$:
	
		a) $S = \emptyset$(Feasible set is empty), constraint $p^* = \infty$
	
		b) $S$ is unbounded \& no minimum, constraint $p^* = -\infty$
\end{itemize}


\begin{figure}
	\centering
	\includegraphics[width=2.1in,height=2.1in]{figures/ch07/figure1012_8.png}
	%\caption{This is an inserted JPG graphic} 
	%\label{fig:graph} 
\end{figure}





"active" constraints are those inequality constraints satisfied with equality. 

Cost decreases if:

\begin{align*}
c^T(x^{(0)} + \bigtriangleup) + d &< c^Tx^{(0)} + d\\
c^T\bigtriangleup &< 0\\
<c, \bigtriangleup> &< 0
\end{align*}


In example constraints 1\&2 are active at optimum.\\

Observation: 

\begin{itemize}
	\item If you are at a vertex(doesn't have to be optimum). 
	
	\item Any "move" that keeps you feasible must be into feasible set $\rightarrow$ opposite vector that define active constraints.
	
	\begin{equation*}
	v - \alpha g^{(1)} - \beta g^{(2)}, \,\,\, \alpha, \beta \geq 0
	\end{equation*}
	
	\item Are these any choices of $\alpha, \beta$ that decrease the cost?
	
	\begin{align*}
	c^T(v - \alpha g^{(1)} - \beta g^{(2)}) + d &\leq c^Tx + d\\
	-\alpha <c, g^{(1)}> - \beta<c, g^{(2)}> &\leq 0
	\end{align*}
\end{itemize}

If:

1) $<c, g^{(1)}> < 0$

2) $<c, g^{(2)}> < 0$

no more into feasible set will decrease the cost.\\ 

Condition for optimality:\\

A feasible vertex $v$: $v\in\{x|Gx \leq h \}$ is an optimal solution to LP with cost $F_0(x) = c^Tx + d$ if $c^Tg^{(i)} < 0$, $\forall i\in$ active set.\\

\begin{figure}
	\centering
	\includegraphics[width=2.1in,height=2.1in]{figures/ch07/figure1012_9.png}
	%\caption{This is an inserted JPG graphic} 
	%\label{fig:graph} 
\end{figure}


Simplex algorithm:

\begin{itemize}
	\item 1) Start ar feasible vertex;
	
	\item 2) Identify direct of cost decrease along an edge;
	
	\item 3) Move on that direction until any further more would violate a previously inactive constraints.
	
	\item 4) Stop + add that new constraint(s) to active set.
	
	\item 5) Repeat
\end{itemize}



%Above are notes for Oct 12

%Below are notes for Oct14



\begin{equation*}
\{x|Gx\leq h \}
\end{equation*}


FIGURE1


If $x =v^{(a)}$, constraints, constraints 1\&2 active, 3 is inactive.

\begin{align*}
<g^{(1)}, v^{(a)}> &= h_1\\
<g^{(2)}, v^{(a)}> &= h_2\\
<g^{(3)}, v^{(a)}> &< h_3
\end{align*}

At $x = v^{(b)}$, constraints 2\&3 active, 1 is inactive.\\

1) Define a vertex $v$:

\begin{itemize}
	\item a) $v\in S$ is a vertex if cannot express $v = \lambda u +(1-\lambda)w$, $u, w\in S$, $\lambda \in [0,1]$
	
	\item b) There exists a cost vector $c\in \Re^n$ s.t. $v$ is unique solution to an LP with cost vector $c$.
	
	\item c) Algebraic characterization of vertices in terms of $A, b, G, h$. $S = \{x|Ax \leq b, Gx\leq h \}$
\end{itemize}



2) Define / Find directions along edges of polyhedron that keep you in the feasible set.

3) How far can you go in direction that decrease cost until "leave" feasible set?

\begin{align*}
min \,\,\, &c^Tx+d\\
s.t. \,\,\, &Ax = b\\
&Gx\leq h
\end{align*}

$x\in \Re^n, c\in \Re^n, d\in \Re, A\in \Re^{q\times n}, b\in \Re^q, G\in \Re^{m\times n}, h\in \Re^m$.

Inequalities:

\begin{align*}
min \,\,\, &c^Tx+d\\
s.t. \,\,\, &Gx\leq h\\
\end{align*}

"Standard Form":

\begin{align*}
min \,\,\, &c^Tx+d\\
s.t. \,\,\, &Ax = b\\
&x\geq 0
\end{align*}

To get to inequality:

\begin{equation*}
Ax = b \Leftrightarrow Ax\geq b, Ax\leq b
\end{equation*}

\begin{align*}
min\,\,\, &c^Tx +d\\
s.t. &\begin{bmatrix}
G\\
A\\
-A
\end{bmatrix} x\leq
\begin{bmatrix}
h\\
b\\
-b
\end{bmatrix}
\end{align*}

Standard Form:

\begin{align*}
min \,\,\, &c^Tx+d\\
s.t. \,\,\, &Gx\leq h\\
&Ax = b
\end{align*}

is equivalent to:

\begin{align*}
min \,\,\, &c^Tx+d\\
s.t. \,\,\, &Gx + s = h\\
&Ax = b\\
&s\geq 0
\end{align*}

is also equivalent to:

\begin{align*}
min \,\,\, &c^T(x^{+} - x^{-})+d\\
s.t. \,\,\, &G(x^{+} - x^{-}) + s = h\\
&A(x^{+} - x^{-}) = b\\
&s\geq 0\\
&x^{+}\geq 0\\
&x^{-}\geq 0
\end{align*}

can come to:

\begin{align*}
min_{x^{+},x^{-},s} &\begin{bmatrix}
c^T - c^T & 0
\end{bmatrix}
\begin{bmatrix}
x^{+}\\
x^{-}\\
s
\end{bmatrix}+d\\
&\begin{bmatrix}
A-A & 0
\end{bmatrix}
\begin{bmatrix}
x^{+}\\
x^{-}\\
s
\end{bmatrix}=b\\
&\begin{bmatrix}
G-G & I
\end{bmatrix}
\begin{bmatrix}
x^{+}\\
x^{-}\\
s
\end{bmatrix}=h\\
&x^{+}\geq 0, x^{-}\geq 0, s\geq 0
\end{align*}




\begin{example}
	\begin{align*}
	&min \Vert Ax - b\Vert_{\infty}\\
	&s.t. Gx \leq h\\
	&\Vert u\Vert_{\infty} = max_{i\in [n]}|u_i|
	\end{align*}
	Introduce helper variable $t$:
	
	FIGURE2
	
	
	\begin{align*}
	min_{x,t} &t\\
	s.t. &Ax - b\leq t\textbf{1}\\
	&Ax - b\leq (-t)\textbf{1}\\
	&Gx \leq h
	\end{align*}
\end{example}


\begin{example}
	FIGURE3
	
	\begin{align*}
	min_x &||Ax - b||_1, \,\,\,\,\, A\in \Re^{q\times n}\\
	s.t. &Gx\leq h
	\end{align*}
	
	\begin{equation*}
	||u||_1 = \sum^q_{i=1} |u_i|
	\end{equation*}
	
	Helper vector $t\in \Re^q$.
	
	\begin{align*}
	min_{x,t} &\sum^q_{i=1}t_i\\
	s.t. \,\,\,\,\,&Gx \leq h\\
	&Ax -b \leq t\\
	&Ax -b \geq -t
	\end{align*}
\end{example}

\begin{example}
	\begin{align*}
	min \,\,\, max_{i\in [q]} &(c^{(i)^T}x + d_i)\\
	s.t. &Gx\leq h
	\end{align*}
	
	\begin{align*}
	min_{x, t\in \Re} &t\\
	s.t. &(c^{(i)^T}x + d_i)\leq t,\,\,\,\,\, \forall i\in [q]\\
	&Gx\leq h
	\end{align*}
	
	FIGURE4
\end{example}



\begin{example}
	Finding the largest $l_2$ ball that fits in a polyhedron $p = \{x| Gx\leq h \}$
	
	FIGURE5
\end{example}

A sphere is fully parameterized by :
\begin{itemize}
	\item Its center $x_c$
	
	\item Its radius $r$
\end{itemize}

A sphere fits in $p$ if:$x_c + u \in p$, $\forall u \,\,\, s.t. ||u||_2\leq r$


\begin{itemize}
	\item $x_c+u \in p$ means that $g^{(i)^T}(x_c + u) \leq h_i,\,\,\, \forall i\in [q]$ \& all $u$ $s.t. ||u||_2\leq r$.
	
	\item Looking at constraints $i$ by itself: $g^{({i})^T}x_c + g^{({i})^T}u \leq h_i$. 
	
	As for $g^{({i})^T}u$, for $||u||_2>r$, what direction should $u$ point in to make left hand side as large as possible?
	
	Set $u=\frac{g^{(1)}}{||g^{(1)}||}r$ 
\end{itemize}

Upshot: if following is satisfied:

\begin{equation*}
g^{(1)^T}[x_c + \frac{g^{(1)}}{||g^{(1)}||_2}r] \leq h_i
\end{equation*}

Then constraint $i$ is satidfied for all $u$ that are of length $r$.

FIGURE6

\begin{align*}
max_{x, r} &r\,\,\,(or\,\,\,min -r)\\
s.t. &g^{(1)^T}x_c + ||g^{(1)}||_r\leq h_i\,\,\,\,\, i\in [q]
\end{align*}

$\rightarrow(x_c, r)\in \Re^{n+1}$

$\rightarrow$ Transformed some quadratic-like problem into an LP.\\


Quadratic program(QP)

\begin{align*}
p^* = min_{x\in \Re^n} &\frac{1}{2}x^THx + c^Tx + d\\
s.t. \,\,\, &Ax = b\\
&Gx \leq h
\end{align*}


%Above are notes for Oct14


\section{10.16}

%Below are notes for Oct16

\begin{align*}
p^* = min_{x\in \Re^n}\,\,\,\,\, &\frac{1}{2}x^THx + c^Tx + d\\
 s.t.\,\,\,\,\, &Ax = b\\
 &Gx\leq h
\end{align*}


FIGUREa

\subsection{Least squares}
\begin{align*}
||Ax - y||_2^2 &= (Ax - y)^T(Ax - y)\\
&= x^TA^TAx - 2y^TAx + y^Ty\\
&= \frac{1}{2}x^T(2A^TA)x - 2y^TAx + ||y||_2^2
\end{align*}

Note: Cannot always manipulate objective of a QP into form of objective of a LS problem.\\

\subsection{Equality constrained QPs}

\begin{align*}
p^* = min_{x\in \Re^n}\,\,\,\,\, &\frac{1}{2}x^THx + c^Tx + d\\
s.t.\,\,\,\,\, &Ax = b
\end{align*}


FIGUREb

\begin{equation*}
\mathcal{A} =\{x | x = \bar{x} + \xi,\,\,\, \text{where } \xi \in N(A) \}
\end{equation*}

Let N be a basis for $N(A)$, can express $\xi \in N(A)$ as $\xi = Nz$ where $z\in \Re^k$, $k =dim(N(A)) \leq n$


Substitute this expression for any feasible $x$ into objective $F_0(\cdot)$


\begin{align*}
F_0(x) &= F_0(\bar{x} + \xi) = F_)(\bar{x} + Nz)\\
&= \frac{1}{2}(\bar{x} + Nz)^TH(\bar{x}+Nz) + c^T(\bar{x} + Nz) + d\\
&= \frac{1}{2}z^T[N^THN]z + [c^TN + \bar{x}^THN]z + [\frac{1}{2}\bar{x}^TH\bar{x} + c^Tx + d] \\
\end{align*}

\begin{equation*}
p^* = min_{z\in \Re^k}\,\,\,\,\, \frac{1}{2}z^T\tilde{H}z + \tilde{c^T}z + \tilde{d}
\end{equation*}

$\rightarrow$ lower-dimensional optimization problem $k =dim(N(A)) \leq n $

$\rightarrow$ still a QP, but an unconstrained QP.



\begin{example}{Markowitz Partfolio optimization}
	"Mean/variance" analysis
	
	\begin{itemize}
		\item Minimize amount of variance in returns for a fixed level of (expected) returns
		
		\item Harry Markowitz(1990 Nobel Prize)
	\end{itemize}
	
	\textbf{Model}: There are n stock, single investment period\\
	
	\textbf{Task}:
	
	\begin{itemize}
		\item Design an investment strategy $p\in \Re^n$
		
		\item $p_i = $ amount/proportion of wealth invested in stock $i$
		
		\item $\sum^n_{i=1}p_i = 1$, "all-in" invest all your money
		
		\item $p_i \geq 0$, "long" positions(no "short time")
		
		\item Wealth normalized to 1
	\end{itemize} 
	
	\textbf{Returns}:
	
	\begin{itemize}
		\item $x\in \Re^n$
		
		\item $x_i$ = return on $i^th$ stock in period if must 1RMB in stock: get $x_i$RMB back.
	\end{itemize} 
	
	
	\textbf{Probabilistic model on returns}:
	
	\begin{itemize}
		\item Expected returns: $\bar{x_i} = \mathbb{E}[x_i]$(known)
		
		\item Your return (random) is $\sum^n_{i=1}p_ix_i = p^Tx$
		
		\item Your expected value $\mathbb{E}[\sum^n_{i=1}p_ix_i] = \sum^n_{i=1}p_i\mathbb{E}[x_i] = \sum^n_{i=1}p_i\bar{x_i} = p^T\bar{x}$
		
		\item variance in returns:
		
		\begin{align*}
		var(p^Tx) &=\mathbb{E}[(p^Tx - p^T\bar{x})^2]\\
		&= \mathbb{E}[(p^T(x - x^T))^2]\\
		&= \mathbb{E}[p^T(x - \bar{x})(x - \bar{x})^Tp]\\
		&= p^T\mathbb{E}[(x - \bar{x})(x - \bar{x})^T]p\\
		&= p^T\Sigma p
		\end{align*}
	\end{itemize}
	
	
	
	
	
	
	
	
	

	
	Given mean returns $\bar{x}$ and covariance matrix of returns $\Sigma \in \Re^{n\times n}$. Design $p$ to minimize "risk" subject to same minimal returns.:
	
	\begin{align*}
	min_{p\in \Re^n}\,\,\,\,\, &p^T\Sigma p\\
	s.t. \,\,\,\,\,&p^T\bar{x}\geq r_{min}\\
	&\textbf{1}^Tp = 1\\
	&p\geq 0
	\end{align*}
\end{example}




\begin{align*}
p^* = min_{x\in \Re^n}\,\,\,\,\, &\frac{1}{2}x^THx + x^Tx + d\\
s.t.\,\,\,\,\, &Ax = b\\
&Gx\leq h
\end{align*}

Discuss geometry of objective

Discuss geometry of feasible set

1) w.l.o.g assume $H\in S^n$, 

\begin{align*}
x^THx &=\frac{1}{2}[x^THx + x^TH^Tx]\\
 &= \frac{1}{2}x^T(H+H^T)x
\end{align*}

Hence always consider $H$ symmetric.\\



\subsection{Symmetric matrices}

1) Eigenvalues: purely real eigenvalues (can sort)

2) Eigenvectors: can always be chosen $\perp$ and can always diagonalize

\begin{equation*}
H = \mathcal{U}\Lambda \mathbb{U}^T  = \sum^n_{i=1}\lambda_iu^{(i)}u^{(i)^T}
\end{equation*}

For 3 cases:

\begin{equation*}
F_0(x) = \frac{1}{2}x^THx + c^Tx + d
\end{equation*}

\begin{itemize}
	\item A) $H\in S^n$ but not PSD
	
	\begin{equation*}
	H = 
	\begin{bmatrix}
	1&0\\
	0&-2
	\end{bmatrix}
	\end{equation*}
	
	\item B) $H\in S^n_+$ but not PD
	\begin{equation*}
	H = 
	\begin{bmatrix}
	1&0\\
	0&0
	\end{bmatrix}
	\end{equation*}
	
	\item C) $H\in S^n_{++}$
	\begin{equation*}
	H = \begin{bmatrix}
	1&0\\
	0&2
	\end{bmatrix}
	\end{equation*}
\end{itemize}



\begin{itemize}
	\item Plot 1: $F_0(x) =\frac{1}{2}x^THx$
	
	\item Plot 2: $F_0(x) =\frac{1}{2}x^THx + [0.5\,\,\, 0.5]x$
	
	\item Plot 3: $F_0(x) =\frac{1}{2}x^THx + [0.5\,\,\, 0]x$
\end{itemize}




MATLAB Graphs may be included later.\\

Case A: 

$H\in S^n$ but $H\notin S^n_+$. (Symmetric not PSD)

There must be an eigenvalue/vector pair $(\lambda, u)$ s.t. $\lambda < 0$.

Set $x_{\alpha} = \alpha u$ for some $\alpha \in \Re$

\begin{align*}
F_0(\alpha u) &= \frac{1}{2}(\alpha u)^TH(\alpha u) + c^T(\alpha u) + d\\
&= \frac{{\alpha}^2}{2} u^T[\sum^n_{i=1}\lambda_i u^{(i)} u^{(i)^T}]u + \alpha c^Tu + d\\
&= \frac{{\alpha}^2}{2}\lambda + \alpha <c,u> + d\\
&= -\frac{{\alpha}^2}{2}|\lambda| + \alpha<c,u> + d\\
&= F(\alpha u)
\end{align*}


Case B: 

$H\in S_+^n$ but $H\notin S^n_{++}$ (PSD not PD)

$\rightarrow$ at least 1 zero eigenvalue.\\

$c \notin R(H)$

$\rightarrow$ There is a complement of $c$ in $R(H)^{\perp} = N(H^T) = N(H)$ and can move in that direction without changing $2^{nd}$ order term while driving $1^{st} $ order term to $\infty$.

$\rightarrow$ Let $c_{||} = \prod_{R(H)}(c)$, $c_{\perp} = \prod_{N(H)}(c)$

$\rightarrow$ $c = c_{||} + c_{\perp}$ orthogonal decomposition lemma(unique decomposition)

$\rightarrow$ Let $x_{\alpha} =-\alpha c_{\perp}$ where $\alpha \in \Re^+$

\begin{align*}
F_0(x_{\alpha}) &=\frac{{\alpha}^2}{2}c_{\perp}^THc_{\perp} + c^T(-\alpha c_{\perp}) + d\\
&= 0 - \alpha(c_{||} + c_{\perp})^Tc_{\perp} + d\\
&= -\alpha(c_{||}^Tc_{\perp} + c_{\perp}^Tc_{\perp}) + d\\
&= -\alpha||c_{\perp}||^2_2 + d
\end{align*}

\begin{itemize}
	\item Case A: $H\notin S^n_+$, $p^* = -\infty$
	
	\item Case B: $H\in S^n_+$ but not $S^n_{++}$, $c\in R(H)$
	
	\item Case C: $H\in S^n_{++}$
\end{itemize}

%Above are notes for Oct16


%\chapter{Convexity}
%\label{ch.convex}
%%% Placeholder for chapter on convexity

%Below are notes for Oct23
\section{Linear/affine/convex/conic hulls \& sets}
Given a set of points $x^{(i)} \in \Re^n$, $i\in [m]$

\begin{equation*}
P = \{x^{(1)}, x^{(2)},..., x^{(m)} \}
\end{equation*}

Consider combinations of form $\sum^m_{i=1} \lambda_i x^{(1)}$

1) The "linear" hull: 

\begin{equation*}
\{x|x = \sum^m_{i=1} \lambda_i x^{(i)}, \lambda_r\in \Re, \forall i\in [m] \}
\end{equation*}

2) The "affine" hull: 

\begin{equation*}
\{x|x = \sum^m_{i=1} \lambda_i x^{(i)}, \lambda_i\in \Re, \sum^n_{i=1}\lambda_i = 1 \}
\end{equation*}


3) The "convex" hull: 

\begin{equation*}
\{x|x = \sum^m_{i=1} \lambda_i x^{(i)}, \lambda_i\in \Re, \lambda_i \geq 0, \sum^m_{i=1}\lambda_i = 1 \}
\end{equation*}


4) The "conic" hull: 

\begin{equation*}
\{x|x = \sum^m_{i=1} \lambda_i x^{(i)}, \lambda_i\in \Re, \lambda_i \geq 0 \}
\end{equation*}


\begin{center}
	\begin{tabular}{|c|c|c|}
	\hline  
   & $\lambda_i \geq 0$   & $\sum^m_{i=1}\lambda_i = 1$ \\
	\hline  
Linear&  no  & no \\
	\hline  
Affine&  no  &yes  \\
	\hline 
Covex&  yes  & yes \\
	\hline  
Conic&  yes  &  no\\
	\hline 
\end{tabular}
\end{center}



1) Linear Hull

Linear hull of $P = span\{x^{(1)},...,x^{(m)} \} =span(P)$

$\rightarrow$ smallest subspace that contains $P$.

2) Affine hull

\begin{align*}
P &= \{x^{(1)}, x^{(2)} \}\\
x &= \lambda_1x^{(1)} + \lambda_2x^{(2)}\\
&= \lambda_1x^{(1)} + (1-\lambda)_1x^{(1)}\\
&= x^{(2)} + \lambda(x^{(1)} - x^{(2)})\\
aff(P) &= x^{(2)} + span(x^{(1)} - x^{(2)})
\end{align*}

\begin{align*}
P &= \{x^{(1)}, x^{(2)}, x^{(3)} \}\\
x &= \lambda_1x^{(1)} + \lambda_2x^{(2)} + \lambda_3x^{(3)}\\
&= (1 - \lambda_2 - \lambda_3)x^{(1)} + \lambda_2x^{(2)} + \lambda_3x^{(3)}\\
&= x^{(1)} + \lambda_2(x^{(2)} - x^{(1)}) + \lambda_3(x^{(3)} - x^{(1)})
\end{align*}

\begin{marginfigure}
	\centering
	\includegraphics[width=1.8in,height=1.8in]{figures/ch07/figure1023_1.png}
	%\caption{This is an inserted JPG graphic} 
	%\label{fig:graph} 
\end{marginfigure}

3) Convex hulls

\begin{align*}
P &= \{x^{(1)},  x^{(2)}\}\\
x &= \lambda_1x^{(1)} + \lambda_2x^{(2)}\\
&= (1-\lambda)x^{(1)} + \lambda x^{(2)}\\
&= x^{(1)} + \lambda(x^{(2)} - x^{(1)})
\end{align*}

\begin{align*}
P &= \{x^{(1)},  x^{(2)}, x^{(3)} \}\\
x &= \lambda_1x^{(1)} + \lambda_2x^{(2)} + \lambda_3x^{(3)}\\
&= x^{(1)} + \lambda_2(x^{(2)} - x^{(1)}) + \lambda_3(x^{(3)} - x^{(1)})\\
&= x^{(1)} + \lambda \gamma(x^{(2)} - x^{(1)}) + (1 - \lambda)\gamma(x^{(3)} - x^{(1)})
\end{align*}


\begin{marginfigure}
	\centering
	\includegraphics[width=1.8in,height=1.8in]{figures/ch07/figure1023_2.png}
	%\caption{This is an inserted JPG graphic} 
	%\label{fig:graph} 
\end{marginfigure}

\begin{marginfigure}
	\centering
	\includegraphics[width=1.8in,height=1.8in]{figures/ch07/figure1023_3.png}
	%\caption{This is an inserted JPG graphic} 
	%\label{fig:graph} 
\end{marginfigure}

4) Conic hulls: $\sum^n_{i=1}\lambda_i x^{(i)}$, $\lambda_i \geq 0$, $\forall i\in [m]$

\begin{align*}
P &= \{x^{(1)}, x^{(2)} \}\\
x &= \lambda_1x^{(1)} + \lambda_2x^{(2)}\\
&= ( \lambda_1 + \lambda_2)[\frac{\lambda_1}{\lambda_1 + \lambda_2}x^{(1)} + \frac{\lambda_2}{\lambda_1 + \lambda_2}x^{(2)}]\\
&= \gamma[\lambda x^{(1)} + (1-\lambda)x^{(2)}]
\end{align*}

\begin{marginfigure}
	\centering
	\includegraphics[width=1.8in,height=1.8in]{figures/ch07/figure1023_4.png}
	%\caption{This is an inserted JPG graphic} 
	%\label{fig:graph} 
\end{marginfigure}

\subsection{Convex Sets}

\begin{definition}{Convex}
	A subset $C\subseteq \Re^n$ is a "convex" set if $\forall x,y \in \mathcal{C}$, then $z\in \mathcal{C}$, $\forall z = \lambda x + (1-\lambda)y$, $\lambda \in [0,1]$
\end{definition}


\begin{definition}{Strictly Convex}
	A convex set is "strictly" convex if $\forall x,y \in \mathcal{C}$, $\forall \lambda \in (0,1)$, $z = \lambda x + (1-\lambda)y \in rel\,int(\mathcal{C})$(relative interior)
\end{definition}

Objects with straight edges are not strictly convex sets.

\begin{definition}{Cone}
	A set $\mathcal{C}\subseteq \Re^n $ is a cone if $\forall x\in \mathcal{C}$, then $\gamma x\in \mathcal{C}, \forall \gamma \geq 0$
\end{definition}

\begin{example}
	1) Hyper-planes are convex: $H = \{x| a^Tx = b \}$, pick $x,y \in H$, is $z =\lambda x + (1-\lambda)y \in H$? $\forall \lambda \in [0,1]$
	
	\begin{align*}
	a^Tz &= a^T(\lambda x + (1-\lambda)y)\\
	&= \lambda(a^Tx) + (1-\lambda)y\\
	&= \lambda b + (1-\lambda)b\\
	&= b
	\end{align*}
	
	
	2) Half-space: $\{x|a^Tx = b \}$, same proof except get inequality here in above equations($a^Tz \neq \lambda b + (1-\lambda)b$).
	
	3) If $c_1, ..., c_n$ al convex sets, then $\mathcal{C} = \cap^m_{i=1}$ is convex.
	
	\begin{proof}
		Pick any $x,y\in \mathcal{C}$ $\Rightarrow$ $x,y\in \mathcal{C}_i,\forall i\in [m]$.
		
		Consider any $z = \lambda x + (1 - \lambda)y$ , is this in $\mathcal{C}$?
		
		Since $x,y \in \mathcal{C}$, $\rightarrow z\in \mathcal{C}_i$ since $\mathcal{C}_i$, $\forall i\in [m]$
		
		If $z\in \mathcal{C}_i$, $\forall i\in [m]$ also intersection $\rightarrow z\in \cap^m_{i=1}C_i = C$
		
	\end{proof}

	In LP $\&$ QP Feasible set:
	
	\begin{equation*}
	\{x|Ax = b \}\cap \{x|Gx\leq b \} = \{\cap^q_{i=1}\{x|a^{(i)^T}x = b_i  \}\cap \{\cap^m_{i=1}\{x|g^{(i)^T}x \leq h_i \}
	\end{equation*}
	
	4) Affine transformations:
	
	If a map $F: \Re^n \rightarrow \Re^m$ is affine ($F(x) = Ax + b$) and $\mathcal{C} \subseteq \Re^n$ is convex, then the image of $\mathcal{C}$ under $F$ is convex.
	
	\begin{equation*}
	F(\mathcal{C}) = \{F(x) | x\in \mathcal{C} \} \subseteq \Re^m
	\end{equation*}

	Also pre-image of a convex set $\tilde{e}$ in $\Re^m$ is convex
	
	\begin{equation*}
	\{x|F(x)\in \mathcal{C} \} \subseteq \Re^n
	\end{equation*}
\end{example}

\begin{marginfigure}
	\centering
	\includegraphics[width=1.8in,height=1.8in]{figures/ch07/figure1023_5.png}
	%\caption{This is an inserted JPG graphic} 
	%\label{fig:graph} 
\end{marginfigure}


Norm balls are convex

\begin{marginfigure}
	\centering
	\includegraphics[width=1.8in,height=1.8in]{figures/ch07/figure1023_6.png}
	%\caption{This is an inserted JPG graphic} 
	%\label{fig:graph} 
\end{marginfigure}

\begin{proof}
	Take any two points $u,v$ s.t. $\Vert u\Vert \leq 1$, $\Vert v\Vert \leq 1$. 
	
	\begin{align*}
	\Vert \lambda u+(1-\lambda)v\Vert &\leq \Vert \lambda u\Vert + \Vert (1-\lambda)v\Vert\\
	&= \vert \lambda \vert \Vert  u\Vert + \vert (1-\lambda) \vert \Vert  v\Vert\\
	&= \lambda\Vert  u\Vert + (1-\lambda)\Vert  v\Vert\\
	&\leq \lambda 1 + (1-\lambda)1\\
	&= 1
	\end{align*}
	
	\begin{equation*}
	\Vert  x\Vert_p = (\sum^n_{i=1}\vert  x_i\vert^p)^{\frac{1}{p}}
	\end{equation*}
	norm if $p\geq 1$
\end{proof}

\subsection{Ellipsoids}

\begin{equation*}
\xi(x_c, P) =\{x|(x - x_c)^TP^{-1}(x - x_c) \leq 1 \}
\end{equation*}
where $P\in S^n_{++}$   is a convex set.

\begin{proof}
	
	$l_2$ norm ball is convex. 
	
	Consider following affine map $F(u) = P^{\frac{1}{2}}u + x_c$.
	
	The set $\{F(u) | \Vert y\Vert_2 \leq 1 \}$ is convex. 
\end{proof}

Define a set:

\begin{align*}
\{F(u) | \Vert u\Vert^2_2 \leq 1 \} &= \{x|x = P^{\frac{1}{2}}u+x_c, \Vert u\Vert^2 \leq 1 \}\\
&= \{x|x - x_c = P^{\frac{1}{2}}u, \Vert u\Vert^2 \leq 1 \}\\
&= \{x|P^{-\frac{1}{2}}(x - x_c) =u, \Vert u\Vert^2 \leq 1 \}\\
&= \{x|\Vert P^{-\frac{1}{2}}(x - x_c)\Vert^2_2 \leq 1 \}\\
&= \{x|(x - x_c)^TP^{-1}(x - x_c) \leq 1 \}
\end{align*}

Intersect these shapes with polyhedron to get feasible set of QCQP.

\begin{example}
	\begin{align*}
	\{x | \Vert Ax - b \Vert^2_2 \leq 1 \} &= \{x | \Vert F(x) \Vert^2 \leq 1 \}  where F(x) = Ax - b
	\end{align*}
	
	Pre-image of convex set under an affine function and so it's convex
\end{example}


%Above are notes for Oct23


%Below are notes for Oct30

%Above are notes for Oct30


%\chapter{First and second order methods}
%\label{ch.algs}
%\input{ch09_algs.tex}


%----------------------------------------------------------------------------------------

\backmatter

%----------------------------------------------------------------------------------------
%	BIBLIOGRAPHY
%----------------------------------------------------------------------------------------

\bibliography{bibliography} % Use the bibliography.bib file for the bibliography
\bibliographystyle{plainnat} % Use the plainnat style of referencing

%----------------------------------------------------------------------------------------

\printindex % Print the index at the very end of the document

\end{document}
